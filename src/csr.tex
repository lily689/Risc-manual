\chapter{``Zicsr'', Control and Status Register (CSR) Instructions, Version 2.0}
\label{csrinsts}

RISC-V defines a separate address space of 4096 Control and Status
registers associated with each hart.  This chapter defines the full
set of CSR instructions that operate on these CSRs.

\begin{commentary}
  While CSRs are primarily used by the privileged architecture, there
  are several uses in unprivileged code including for counters and
  timers, and for floating-point status.

  The counters and timers are no longer considered mandatory parts of
  the standard base ISAs, and so the CSR instructions required to
  access them have been moved out of the base ISA chapter into this
  separate chapter.
\end{commentary}

\section{CSR Instructions}

All CSR instructions atomically read-modify-write a single CSR, whose
CSR specifier is encoded in the 12-bit {\em csr} field of the
instruction held in bits 31--20.  The immediate forms use a 5-bit
zero-extended immediate encoded in the {\em rs1} field.

\vspace{-0.2in}
\begin{center}
\begin{tabular}{M@{}R@{}F@{}R@{}S}
\\
\instbitrange{31}{20} &
\instbitrange{19}{15} &
\instbitrange{14}{12} &
\instbitrange{11}{7} &
\instbitrange{6}{0} \\
\hline
\multicolumn{1}{|c|}{csr} &
\multicolumn{1}{c|}{rs1} &
\multicolumn{1}{c|}{funct3} &
\multicolumn{1}{c|}{rd} &
\multicolumn{1}{c|}{opcode} \\
\hline
12 & 5 & 3 & 5 & 7 \\
source/dest  & source & CSRRW  & dest & SYSTEM \\
source/dest  & source & CSRRS  & dest & SYSTEM \\
source/dest  & source & CSRRC  & dest & SYSTEM \\
source/dest  & uimm[4:0]   & CSRRWI & dest & SYSTEM \\
source/dest  & uimm[4:0]   & CSRRSI & dest & SYSTEM \\
source/dest  & uimm[4:0]   & CSRRCI & dest & SYSTEM \\
\end{tabular}
\end{center}

The CSRRW (Atomic Read/Write CSR) instruction atomically swaps values
in the CSRs and integer registers. CSRRW reads the old value of the
CSR, zero-extends the value to XLEN bits, then writes it to integer
register {\em rd}.  The initial value in {\em rs1} is written to the
CSR.  If {\em rd}={\tt x0}, then the instruction shall not read the CSR
and shall not cause any of the side effects that might occur on a CSR
read.

The CSRRS (Atomic Read and Set Bits in CSR) instruction reads the
value of the CSR, zero-extends the value to XLEN bits, and writes it
to integer register {\em rd}.  The initial value in integer register
{\em rs1} is treated as a bit mask that specifies bit positions to be
set in the CSR.  Any bit that is high in {\em rs1} will cause the
corresponding bit to be set in the CSR, if that CSR bit is writable.
Other bits in the CSR are unaffected (though CSRs might have side
effects when written).

The CSRRC (Atomic Read and Clear Bits in CSR) instruction reads the
value of the CSR, zero-extends the value to XLEN bits, and writes it
to integer register {\em rd}.  The initial value in integer register
{\em rs1} is treated as a bit mask that specifies bit positions to be
cleared in the CSR.  Any bit that is high in {\em rs1} will cause the
corresponding bit to be cleared in the CSR, if that CSR bit is
writable.  Other bits in the CSR are unaffected.

For both CSRRS and CSRRC, if {\em rs1}={\tt x0}, then the instruction
will not write to the CSR at all, and so shall not cause any of the
side effects that might otherwise occur on a CSR write, nor
raise illegal instruction exceptions on accesses to read-only CSRs.
Both CSRRS and CSRRC always read the addressed CSR and cause any read
side effects regardless of {\em rs1} and {\em rd} fields.  Note that
if {\em rs1} specifies a register holding a zero value other than {\tt
  x0}, the instruction will still attempt to write the unmodified
value back to the CSR and will cause any attendant side effects.  A
CSRRW with {\em rs1}={\tt x0} will attempt to write zero to the
destination CSR.

The CSRRWI, CSRRSI, and CSRRCI variants are similar to CSRRW, CSRRS,
and CSRRC respectively, except they update the CSR using an XLEN-bit
value obtained by zero-extending a 5-bit unsigned immediate (uimm[4:0]) field
encoded in the {\em rs1} field instead of a value from an integer
register.  For CSRRSI and CSRRCI, if the uimm[4:0] field is zero, then
these instructions will not write to the CSR, and shall not cause any
of the side effects that might otherwise occur on a CSR write, nor raise
illegal instruction exceptions on accesses to read-only CSRs.
For CSRRWI, if {\em rd}={\tt x0}, then the instruction shall not read the
CSR and shall not cause any of the side effects that might occur on a
CSR read.  Both CSRRSI and CSRRCI will always read the CSR and cause
any read side effects regardless of {\em rd} and {\em rs1} fields.

\begin{table}
  \centering
  \begin{tabular}{|l|r|r|c|c|}
    \hline
    \multicolumn{5}{|c|}{Register operand} \\
    \hline
  Instruction &  rd &  rs1 & read CSR? & write CSR? \\
    \hline
    CSRRW     &  {\tt x0} &   -        &       no  &       yes \\
    CSRRW     & !{\tt x0} &   -        &      yes  &       yes \\
    CSRR{S/C} &        -  &   {\tt x0} &       yes &       no \\
    CSRR{S/C} &       -   &  !{\tt x0} &       yes &      yes \\
    \hline
    \multicolumn{5}{|c|}{Immediate operand} \\
    \hline
Instruction & rd   & uimm & read CSR? & write CSR? \\
    \hline
   CSRRWI   &   {\tt x0} &  -   &     no    &      yes    \\
   CSRRWI   &  !{\tt x0} &  -   &    yes    &      yes    \\ 
 CSRR{S/C}I &       -    &  0   &    yes    &       no    \\
 CSRR{S/C}I &       -    & !0   &    yes    &      yes    \\
 \hline
  \end{tabular}
  \caption{Table showing whether a CSR instruction reads or writes a
    given CSR.  The CSRRS and CSRRC instructions have same behavior so
  are shown as CSRR{S/C} in Table.}
  \label{tab:csrsideeffects}
\end{table}

Table~\ref{tab:csrsideeffects} summarizes the behavior of the CSR
instructions with respect to whether they read and/or write the CSR.

For any event or consequence that occurs due to a CSR having a particular
value, if a write to the CSR gives it that value, the resulting event or
consequence is said to be an \emph{indirect effect} of the write.
Indirect effects of a CSR write are not considered by the RISC-V ISA to
be side effects of that write.

\begin{commentary}
  The CSRs defined so far in this volume
  do not have any architectural side effects on reads or writes.
  Custom extensions might add CSRs for which accesses have side effects.
\end{commentary}

Some CSRs, such as the instructions-retired counter, {\tt instret},
may be modified as side effects of instruction execution.  In these
cases, if a CSR access instruction reads a CSR, it reads the value
prior to the execution of the instruction.  If a CSR access
instruction writes such a CSR, the write is done instead of the
increment.  In particular, a value written to {\tt instret} by one
instruction will be the value read by the following instruction.

The assembler pseudoinstruction to read a CSR, CSRR {\em rd, csr}, is
encoded as CSRRS {\em rd, csr, x0}.  The assembler pseudoinstruction
to write a CSR, CSRW {\em csr, rs1}, is encoded as CSRRW {\em x0, csr,
  rs1}, while CSRWI {\em csr, uimm}, is encoded as CSRRWI {\em x0,
  csr, uimm}.

Further assembler pseudoinstructions are defined to set and clear
bits in the CSR when the old value is not required: CSRS/CSRC {\em
  csr, rs1}; CSRSI/CSRCI {\em csr, uimm}.


\subsection*{CSR Access Ordering}

On a given hart, explicit and implicit CSR access are performed in program
order with respect to those instructions whose execution behavior is affected
by the state of the accessed CSR. In particular, a CSR access is performed
after the execution of any prior instructions in program order whose behavior
modifies or is modified by the CSR state and before the execution of any
subsequent instructions in program order whose behavior modifies or is modified
by the CSR state. Furthermore, a CSR read access instruction returns the
accessed CSR state before the execution of the instruction, while a CSR write
access instruction updates the accessed CSR state after the execution of the
instruction.

Where the above program order does not hold, CSR accesses are weakly ordered,
and the local hart or other harts may observe the CSR accesses in an order
different from program order. In addition, CSR accesses are not ordered with
respect to explicit memory accesses, unless a CSR access modifies the execution
behavior of the instruction that performs the explicit memory access or unless
a CSR access and an explicit memory access are ordered by either the syntactic
dependencies defined by the memory model or the ordering requirements defined
by the Memory-Ordering PMAs section in Volume II of this manual. To enforce
ordering in all other cases, software should execute a FENCE instruction
between the relevant accesses. For the purposes of the FENCE instruction, CSR
read accesses are classified as device input (I), and CSR write accesses are
classified as device output (O).

\begin{commentary}
Informally, the CSR space acts as a weakly ordered memory-mapped I/O region, as
defined by the Memory-Ordering PMAs section in Volume II of this manual. As a
result, the order of CSR accesses with respect to all other accesses is
constrained by the same mechanisms that constrain the order of memory-mapped
I/O accesses to such a region.

These CSR-ordering constraints are imposed primarily to support ordering main
memory and memory-mapped I/O accesses with respect to reads of the {\tt time}
CSR.  With the exception of the {\tt time}, {\tt cycle}, and {\tt mcycle} CSRs,
the CSRs defined thus far in Volumes I and II of this specification are not
directly accessible to other harts or devices and cause no side effects visible
to other harts or devices.  Thus, accesses to CSRs other than the
aforementioned three can be freely reordered with respect to FENCE instructions
without violating this specification.
\end{commentary}

For CSR accesses that cause side effects, the above ordering constraints apply
to the order of the initiation of those side effects but does not necessarily
apply to the order of the completion of those side effects.

The hardware platform may define that accesses to certain CSRs are
strongly ordered, as defined by the Memory-Ordering PMAs section in Volume II
of this manual. Accesses to strongly ordered CSRs have stronger ordering
constraints with respect to accesses to both weakly ordered CSRs and accesses
to memory-mapped I/O regions.
