\chapter{Extending RISC-V}
\label{extensions}

In addition to supporting standard general-purpose software
development, another goal of RISC-V is to provide a basis for more
specialized instruction set extensions or more customized
accelerators.  The instruction encoding spaces and optional
variable-length instruction encoding are designed to make it easier to
leverage software development effort for the standard ISA toolchain
when building more customized processors.  For example, the intent is
to continue to provide full software support for implementations that
only use the standard I base, perhaps together with many non-standard
instruction set extensions.

This chapter describes various ways in which the base RISC-V ISA can
be extended, together with the scheme for managing instruction set
extensions developed by independent groups.  This volume only deals
with the user-level ISA, although the same approach and terminology is
used for supervisor-level extensions described in the second volume.

\section{Extension Terminology}

This section defines some standard terminology for describing RISC-V
extensions.
\vspace{-0.2in}
\subsection*{Standard versus Non-Standard Extension}

Any RISC-V processor implementation must support a base integer ISA
(RV32I or RV64I).  In addition, an implementation may support one or
more extensions.  We divide extensions into two broad categories: {\em
  standard} versus {\em non-standard}.
\begin{itemize}
\item A standard extension is one that is generally useful and that is
  designed to not conflict with any other standard extension.
  Currently, ``MAFDQLCBTPV'', described in other chapters of this
  manual, are either complete or planned standard extensions.
\item A non-standard extension may be highly specialized and may
  conflict with other standard or non-standard extensions.  We
  anticipate a wide variety of non-standard extensions will be
  developed over time, with some eventually being promoted to standard
  extensions.
\end{itemize}

\vspace{-0.2in}
\subsection*{Instruction Encoding Spaces and Prefixes}

An instruction encoding space is some number of instruction bits
within which a base ISA or ISA extension is encoded.  RISC-V supports
varying instruction lengths, but even within a single instruction
length, there are various sizes of encoding space available.  For
example, the base ISA is defined within a 30-bit encoding space (bits
31--2 of the 32-bit instruction), while the atomic extension ``A''
fits within a 25-bit encoding space (bits 31--7).

We use the term {\em prefix} to refer to the bits to the {\em right}
of an instruction encoding space (since RISC-V is little-endian, the
bits to the right are stored at earlier memory addresses, hence form a
prefix in instruction-fetch order).  The prefix for the standard base
ISA encoding is the two-bit ``11'' field held in bits 1--0 of the
32-bit word, while the prefix for the standard atomic extension ``A''
is the seven-bit ``0101111'' field held in bits 6--0 of the 32-bit
word representing the AMO major opcode.  A quirk of the encoding
format is that the 3-bit funct3 field used to encode a minor opcode is
not contiguous with the major opcode bits in the 32-bit instruction
format, but is considered part of the prefix for 22-bit instruction
spaces.

Although an instruction encoding space could be of any size, adopting
a smaller set of common sizes simplifies packing independently
developed extensions into a single global encoding.
Table~\ref{encodingspaces} gives the suggested sizes for RISC-V.

\begin{table}[H]
\begin{center}
\begin{tabular}{|c|l|r|r|r|r|}
\hline
\multicolumn{1}{|c|}{Size} & \multicolumn{1}{|c|}{Usage} &
\multicolumn{4}{|c|}{\# Available in standard instruction length} \\ \cline{3-6}
 & &
\multicolumn{1}{|c|}{16-bit} &
\multicolumn{1}{|c|}{32-bit} &
\multicolumn{1}{|c|}{48-bit} &
\multicolumn{1}{|c|}{64-bit} \\ \hline \hline
14-bit & Quadrant of compressed 16-bit encoding & 3       &         &         &         \\ \hline \hline
22-bit & Minor opcode in base 32-bit encoding   &         & $2^{8}$ & $2^{20}$ & $2^{35}$ \\ \hline
25-bit & Major opcode in base 32-bit encoding   &         &      32 & $2^{17}$ & $2^{32}$ \\ \hline
30-bit & Quadrant of base 32-bit encoding       &         &       1 & $2^{12}$ & $2^{27}$ \\ \hline \hline
32-bit & Minor opcode in 48-bit encoding        &         &         & $2^{10}$ & $2^{25}$ \\ \hline
37-bit & Major opcode in 48-bit encoding        &         &         &       32 & $2^{20}$ \\ \hline
40-bit & Quadrant of 48-bit encoding            &         &         &        4 & $2^{17}$ \\ \hline \hline
45-bit & Sub-minor opcode in 64-bit encoding    &         &         &          & $2^{12}$ \\ \hline
48-bit & Minor opcode in 64-bit encoding        &         &         &          & $2^{9}$  \\ \hline
52-bit & Major opcode in 64-bit encoding        &         &         &          &      32\\ \hline
\end{tabular}
\end{center}
\caption{Suggested standard RISC-V instruction encoding space sizes.}
\label{encodingspaces}
\end{table}

\vspace{-0.2in}
\subsection*{Greenfield versus Brownfield Extensions}

We use the term {\em greenfield extension} to describe an extension
that begins populating a new instruction encoding space, and hence can
only cause encoding conflicts at the prefix level.  We use the term
{\em brownfield extension} to describe an extension that fits around
existing encodings in a previously defined instruction space.  A
brownfield extension is necessarily tied to a particular greenfield
parent encoding, and there may be multiple brownfield extensions to
the same greenfield parent encoding.  For example, the base ISAs are
greenfield encodings of a 30-bit instruction space, while the FDQ
floating-point extensions are all brownfield extensions adding to the
parent base ISA 30-bit encoding space.

Note that we consider the standard A extension to have a greenfield
encoding as it defines a new previously empty 25-bit encoding space in
the leftmost bits of the full 32-bit base instruction encoding, even
though its standard prefix locates it within the 30-bit encoding space
of the base ISA.  Changing only its single 7-bit prefix could move the
A extension to a different 30-bit encoding space while only worrying
about conflicts at the prefix level, not within the encoding space
itself.

\begin{table}[H]
{
\begin{center}
\begin{tabular}{|r|c|c|}
\hline
 & Adds state & No new state \\ \hline
Greenfield & RV32I(30), RV64I(30) & A(25) \\\hline
Brownfield & F(I), D(F), Q(D) & M(I) \\
\hline
\end{tabular}
\end{center}
}
\caption{Two-dimensional characterization of standard instruction set
  extensions.}
\label{exttax}
\end{table}

Table~\ref{exttax} shows the bases and standard extensions placed in a
simple two-dimensional taxonomy.  One axis is whether the extension is
greenfield or brownfield, while the other axis is whether the
extension adds architectural state.  For greenfield extensions, the
size of the instruction encoding space is given in parentheses.  For
brownfield extensions, the name of the extension (greenfield or
brownfield) it builds upon is given in parentheses.  Additional
user-level architectural state usually implies changes to the
supervisor-level system or possibly to the standard calling
convention.

Note that RV64I is not considered an extension of RV32I, but a
different complete base encoding.

\vspace{-0.2in}
\subsection*{Standard-Compatible Global Encodings}

A complete or {\em global} encoding of an ISA for an actual RISC-V
implementation must allocate a unique non-conflicting prefix for every
included instruction encoding space.  The bases and every standard
extension have each had a standard prefix allocated to ensure they can
all coexist in a global encoding.

A {\em standard-compatible} global encoding is one where the base and
every included standard extension have their standard prefixes.  A
standard-compatible global encoding can include non-standard
extensions that do not conflict with the included standard extensions.
A standard-compatible global encoding can also use standard prefixes
for non-standard extensions if the associated standard extensions are
not included in the global encoding.  In other words, a standard
extension must use its standard prefix if included in a
standard-compatible global encoding, but otherwise its prefix is free
to be reallocated.  These constraints allow a common toolchain to
target the standard subset of any RISC-V standard-compatible global
encoding.

\vspace{-0.2in}
\subsection*{Guaranteed Non-Standard Encoding Space}

To support development of proprietary custom extensions, portions of
the encoding space are guaranteed to never be used by standard
extensions.

\section{RISC-V Extension Design Philosophy}

We intend to support a large number of independently developed
extensions by encouraging extension developers to operate within
instruction encoding spaces, and by providing tools to pack these into
a standard-compatible global encoding by allocating unique prefixes.
Some extensions are more naturally implemented as brownfield
augmentations of existing extensions, and will share whatever prefix
is allocated to their parent greenfield extension.  The standard
extension prefixes avoid spurious incompatibilities in the encoding of
core functionality, while allowing custom packing of more esoteric
extensions.

This capability of repacking RISC-V extensions into different
standard-compatible global encodings can be used in a number of ways.

One use-case is developing highly specialized custom accelerators,
designed to run kernels from important application domains.  These
might want to drop all but the base integer ISA and add in only the
extensions that are required for the task in hand.  The base ISA has
been designed to place minimal requirements on a hardware
implementation, and has been encoded to use only a small fraction of a
32-bit instruction encoding space.

Another use-case is to build a research prototype for a new type of
instruction set extension.  The researchers might not want to expend
the effort to implement a variable-length instruction-fetch unit, and
so would like to prototype their extension using a simple 32-bit
fixed-width instruction encoding.  However, this new extension might
be too large to coexist with standard extensions in the 32-bit space.
If the research experiments do not need all of the standard
extensions, a standard-compatible global encoding might drop the
unused standard extensions and reuse their prefixes to place the
proposed extension in a non-standard location to simplify engineering
of the research prototype.  Standard tools will still be able to
target the base and any standard extensions that are present to reduce
development time.  Once the instruction set extension has been
evaluated and refined, it could then be made available for packing
into a larger variable-length encoding space to avoid conflicts with
all standard extensions.

The following sections describe increasingly sophisticated strategies
for developing implementations with new instruction set extensions.
These are mostly intended for use in highly customized, educational,
or experimental architectures rather than for the main line of RISC-V
ISA development.

\section{Extensions within fixed-width 32-bit instruction format}
\label{fix32b}

In this section, we discuss adding extensions to implementations that
only support the base fixed-width 32-bit instruction format.

\begin{commentary}
We anticipate the simplest fixed-width 32-bit encoding will be popular for
many restricted accelerators and research prototypes.
\end{commentary}

\subsection*{Available 30-bit instruction encoding spaces}

In the standard encoding, three of the available 30-bit instruction
encoding spaces (those with 2-bit prefixes 00, 01, and 10) are used to
enable the optional compressed instruction extension.  However, if the
compressed instruction set extension is not required, then these three
further 30-bit encoding spaces become available.  This quadruples the
available encoding space within the 32-bit format.

\subsection*{Available 25-bit instruction encoding spaces}

A 25-bit instruction encoding space corresponds to a major opcode in
the base and standard extension encodings.

There are four major opcodes expressly reserved for custom extensions
(Table~\ref{opcodemap}), each of which represents a 25-bit encoding
space.  Two of these are reserved for eventual use in the RV128 base
encoding (will be OP-IMM-64 and OP-64), but can be used for standard
or non-standard extensions for RV32 and RV64.

The two opcodes reserved for RV64 (OP-IMM-32 and OP-32) can also be
used for standard and non-standard extensions to RV32 only.

If an implementation does not require floating-point, then the seven
major opcodes reserved for standard floating-point extensions
(LOAD-FP, STORE-FP, MADD, MSUB, NMSUB, NMADD, OP-FP) can be reused for
non-standard extensions.  Similarly, the AMO major opcode can be
reused if the standard atomic extensions are not required.

If an implementation does not require instructions longer than
32-bits, then an additional four major opcodes are available (those
marked in gray in Table~\ref{opcodemap}).

The base RV32I encoding uses only 11 major opcodes plus 3 reserved
opcodes, leaving up to 18 available for extensions.  The base RV64I
encoding uses only 13 major opcodes plus 3 reserved opcodes, leaving
up to 16 available for extensions.

\subsection*{Available 22-bit instruction encoding spaces}

A 22-bit encoding space corresponds to a funct3 minor opcode space in
the base and standard extension encodings.  Several major opcodes have
a funct3 field minor opcode that is not completely occupied, leaving
available several 22-bit encoding spaces.

Usually a major opcode selects the format used to encode operands in
the remaining bits of the instruction, and ideally, an extension
should follow the operand format of the major opcode to simplify
hardware decoding.

\subsection*{Other spaces}

Smaller spaces are available under certain major opcodes, and not all
minor opcodes are entirely filled.

\section{Adding aligned 64-bit instruction extensions}

The simplest approach to provide space for extensions that are too
large for the base 32-bit fixed-width instruction format is to add
naturally aligned 64-bit instructions.  The implementation must still
support the 32-bit base instruction format, but can require that
64-bit instructions are aligned on 64-bit boundaries to simplify
instruction fetch, with a 32-bit NOP instruction used as alignment
padding where necessary.

To simplify use of standard tools, the 64-bit instructions should be
encoded as described in Figure~\ref{instlengthcode}.  However, an
implementation might choose a non-standard instruction-length encoding
for 64-bit instructions, while retaining the standard encoding for
32-bit instructions.  For example, if compressed instructions are not
required, then a 64-bit instruction could be encoded using one or more
zero bits in the first two bits of an instruction.

\begin{commentary}
We anticipate processor generators that produce instruction-fetch
units capable of automatically handling any combination of supported
variable-length instruction encodings.
\end{commentary}

\section{Supporting VLIW encodings}

Although RISC-V was not designed as a base for a pure VLIW machine,
VLIW encodings can be added as extensions using several alternative
approaches. In all cases, the base 32-bit encoding has to be supported
to allow use of any standard software tools.

\subsection*{Fixed-size instruction group}

The simplest approach is to define a single large naturally aligned
instruction format (e.g., 128 bits) within which VLIW operations are
encoded.  In a conventional VLIW, this approach would tend to waste
instruction memory to hold NOPs, but a RISC-V-compatible
implementation would have to also support the base 32-bit
instructions, confining the VLIW code size expansion to
VLIW-accelerated functions.

\subsection*{Encoded-Length Groups}

Another approach is to use the standard length encoding from
Figure~\ref{instlengthcode} to encode parallel instruction groups,
allowing NOPs to be compressed out of the VLIW instruction.  For
example, a 64-bit instruction could hold two 28-bit operations, while
a 96-bit instruction could hold three 28-bit operations, and so on.
Alternatively, a 48-bit instruction could hold one 42-bit operation,
while a 96-bit instruction could hold two 42-bit operations, and so
on.

This approach has the advantage of retaining the base ISA encoding for
instructions holding a single operation, but has the disadvantage of
requiring a new 28-bit or 42-bit encoding for operations within the
VLIW instructions, and misaligned instruction fetch for larger groups.
One simplification is to not allow VLIW instructions to straddle
certain microarchitecturally significant boundaries (e.g., cache lines
or virtual memory pages).

\subsection*{Fixed-Size Instruction Bundles}

Another approach, similar to Itanium, is to use a larger naturally
aligned fixed instruction bundle size (e.g., 128 bits) across which
parallel operation groups are encoded.  This simplifies instruction
fetch, but shifts the complexity to the group execution engine.  To
remain RISC-V compatible, the base 32-bit instruction would still have
to be supported.

\subsection*{End-of-Group bits in Prefix}

None of the above approaches retains the RISC-V encoding for the
individual operations within a VLIW instruction.  Yet another approach
is to repurpose the two prefix bits in the fixed-width 32-bit
encoding.  One prefix bit can be used to signal ``end-of-group'' if
set, while the second bit could indicate execution under a predicate
if clear.  Standard RISC-V 32-bit instructions generated by tools
unaware of the VLIW extension would have both prefix bits set (11) and
thus have the correct semantics, with each instruction at the end of a
group and not predicated.

The main disadvantage of this approach is that the base ISA lacks the
complex predication support usually required in an aggressive VLIW
system, and it is difficult to add space to specify more predicate
registers in the standard 30-bit encoding space.
