\chapter{RV32/64G Instruction Set Listings}

One goal of the RISC-V project is that it be used as a stable software
development target.  For this purpose, we define a combination of a
base ISA (RV32I or RV64I) plus selected standard extensions (IMAFD) as
a ``general-purpose'' ISA, and we use the abbreviation G for the IMAFD
combination of instruction set extensions.    This chapter presents
opcode maps and instruction set listings for RV32G and RV64G.

\input{opcode-map}

Table~\ref{opcodemap} shows a map of the major opcodes for RVG.  Major
opcodes with 3 or more lower bits set are reserved for instruction
lengths greater than 32 bits.  Opcodes marked as {\em reserved} should
be avoided for custom instruction set extensions as they might be used
by future standard extensions.  Major opcodes marked as {\em custom-0}
and {\em custom-1} will be avoided by future standard extensions and
are recommended for use by custom instruction set extensions within
the base 32-bit instruction format.  The opcodes marked {\em
  custom-2/rv128} and {\em custom-3/rv128} are reserved for future use
by RV128, but will otherwise be avoided for standard extensions and so
can also be used for custom instruction set extensions in RV32 and
RV64.

We believe RV32G and RV64G provide simple but complete instruction
sets for a broad range of general-purpose computing.  The optional
compressed instruction set described in Chapter~\ref{compressed} can
be added (forming RV32GC and RV64GC) to improve performance, code
size, and energy efficiency, though with some additional hardware
complexity.

As we move beyond IMAFDC into further instruction set extensions, the
added instructions tend to be more domain-specific and only provide
benefits to a restricted class of applications, e.g., for multimedia
or security.  Unlike most commercial ISAs, the RISC-V ISA design
clearly separates the base ISA and broadly applicable standard
extensions from these more specialized additions.
Chapter~\ref{extensions} has a more extensive discussion of ways to
add extensions to the RISC-V ISA.


\newpage

\begin{table}[p]
\begin{small}
\begin{center}
\begin{tabular}{p{0in}p{0.4in}p{0.05in}p{0.05in}p{0.05in}p{0.05in}p{0.4in}p{0.6in}p{0.4in}p{0.6in}p{0.7in}l}
& & & & & & & & & & \\
                      &
\multicolumn{1}{l}{\instbit{31}} &
\multicolumn{1}{r}{\instbit{27}} &
\instbit{26} &
\instbit{25} &
\multicolumn{1}{l}{\instbit{24}} &
\multicolumn{1}{r}{\instbit{20}} &
\instbitrange{19}{15} &
\instbitrange{14}{12} &
\instbitrange{11}{7} &
\instbitrange{6}{0} \\
\cline{2-11}


&
\multicolumn{4}{|c|}{funct7} &
\multicolumn{2}{c|}{rs2} &
\multicolumn{1}{c|}{rs1} &
\multicolumn{1}{c|}{funct3} &
\multicolumn{1}{c|}{rd} &
\multicolumn{1}{c|}{opcode} & R-type \\
\cline{2-11}


&
\multicolumn{6}{|c|}{imm[11:0]} &
\multicolumn{1}{c|}{rs1} &
\multicolumn{1}{c|}{funct3} &
\multicolumn{1}{c|}{rd} &
\multicolumn{1}{c|}{opcode} & I-type \\
\cline{2-11}


&
\multicolumn{4}{|c|}{imm[11:5]} &
\multicolumn{2}{c|}{rs2} &
\multicolumn{1}{c|}{rs1} &
\multicolumn{1}{c|}{funct3} &
\multicolumn{1}{c|}{imm[4:0]} &
\multicolumn{1}{c|}{opcode} & S-type \\
\cline{2-11}


&
\multicolumn{4}{|c|}{imm[12$\vert$10:5]} &
\multicolumn{2}{c|}{rs2} &
\multicolumn{1}{c|}{rs1} &
\multicolumn{1}{c|}{funct3} &
\multicolumn{1}{c|}{imm[4:1$\vert$11]} &
\multicolumn{1}{c|}{opcode} & B-type \\
\cline{2-11}


&
\multicolumn{8}{|c|}{imm[31:12]} &
\multicolumn{1}{c|}{rd} &
\multicolumn{1}{c|}{opcode} & U-type \\
\cline{2-11}


&
\multicolumn{8}{|c|}{imm[20$\vert$10:1$\vert$11$\vert$19:12]} &
\multicolumn{1}{c|}{rd} &
\multicolumn{1}{c|}{opcode} & J-type \\
\cline{2-11}


&
\multicolumn{10}{c}{} & \\
&
\multicolumn{10}{c}{\bf RV32I Base Instruction Set} & \\
\cline{2-11}
  

&
\multicolumn{8}{|c|}{imm[31:12]} &
\multicolumn{1}{c|}{rd} &
\multicolumn{1}{c|}{0110111} & LUI \\
\cline{2-11}
  

&
\multicolumn{8}{|c|}{imm[31:12]} &
\multicolumn{1}{c|}{rd} &
\multicolumn{1}{c|}{0010111} & AUIPC \\
\cline{2-11}
  

&
\multicolumn{8}{|c|}{imm[20$\vert$10:1$\vert$11$\vert$19:12]} &
\multicolumn{1}{c|}{rd} &
\multicolumn{1}{c|}{1101111} & JAL \\
\cline{2-11}
  

&
\multicolumn{6}{|c|}{imm[11:0]} &
\multicolumn{1}{c|}{rs1} &
\multicolumn{1}{c|}{000} &
\multicolumn{1}{c|}{rd} &
\multicolumn{1}{c|}{1100111} & JALR \\
\cline{2-11}
  

&
\multicolumn{4}{|c|}{imm[12$\vert$10:5]} &
\multicolumn{2}{c|}{rs2} &
\multicolumn{1}{c|}{rs1} &
\multicolumn{1}{c|}{000} &
\multicolumn{1}{c|}{imm[4:1$\vert$11]} &
\multicolumn{1}{c|}{1100011} & BEQ \\
\cline{2-11}
  

&
\multicolumn{4}{|c|}{imm[12$\vert$10:5]} &
\multicolumn{2}{c|}{rs2} &
\multicolumn{1}{c|}{rs1} &
\multicolumn{1}{c|}{001} &
\multicolumn{1}{c|}{imm[4:1$\vert$11]} &
\multicolumn{1}{c|}{1100011} & BNE \\
\cline{2-11}
  

&
\multicolumn{4}{|c|}{imm[12$\vert$10:5]} &
\multicolumn{2}{c|}{rs2} &
\multicolumn{1}{c|}{rs1} &
\multicolumn{1}{c|}{100} &
\multicolumn{1}{c|}{imm[4:1$\vert$11]} &
\multicolumn{1}{c|}{1100011} & BLT \\
\cline{2-11}
  

&
\multicolumn{4}{|c|}{imm[12$\vert$10:5]} &
\multicolumn{2}{c|}{rs2} &
\multicolumn{1}{c|}{rs1} &
\multicolumn{1}{c|}{101} &
\multicolumn{1}{c|}{imm[4:1$\vert$11]} &
\multicolumn{1}{c|}{1100011} & BGE \\
\cline{2-11}
  

&
\multicolumn{4}{|c|}{imm[12$\vert$10:5]} &
\multicolumn{2}{c|}{rs2} &
\multicolumn{1}{c|}{rs1} &
\multicolumn{1}{c|}{110} &
\multicolumn{1}{c|}{imm[4:1$\vert$11]} &
\multicolumn{1}{c|}{1100011} & BLTU \\
\cline{2-11}
  

&
\multicolumn{4}{|c|}{imm[12$\vert$10:5]} &
\multicolumn{2}{c|}{rs2} &
\multicolumn{1}{c|}{rs1} &
\multicolumn{1}{c|}{111} &
\multicolumn{1}{c|}{imm[4:1$\vert$11]} &
\multicolumn{1}{c|}{1100011} & BGEU \\
\cline{2-11}
  

&
\multicolumn{6}{|c|}{imm[11:0]} &
\multicolumn{1}{c|}{rs1} &
\multicolumn{1}{c|}{000} &
\multicolumn{1}{c|}{rd} &
\multicolumn{1}{c|}{0000011} & LB \\
\cline{2-11}
  

&
\multicolumn{6}{|c|}{imm[11:0]} &
\multicolumn{1}{c|}{rs1} &
\multicolumn{1}{c|}{001} &
\multicolumn{1}{c|}{rd} &
\multicolumn{1}{c|}{0000011} & LH \\
\cline{2-11}
  

&
\multicolumn{6}{|c|}{imm[11:0]} &
\multicolumn{1}{c|}{rs1} &
\multicolumn{1}{c|}{010} &
\multicolumn{1}{c|}{rd} &
\multicolumn{1}{c|}{0000011} & LW \\
\cline{2-11}
  

&
\multicolumn{6}{|c|}{imm[11:0]} &
\multicolumn{1}{c|}{rs1} &
\multicolumn{1}{c|}{100} &
\multicolumn{1}{c|}{rd} &
\multicolumn{1}{c|}{0000011} & LBU \\
\cline{2-11}
  

&
\multicolumn{6}{|c|}{imm[11:0]} &
\multicolumn{1}{c|}{rs1} &
\multicolumn{1}{c|}{101} &
\multicolumn{1}{c|}{rd} &
\multicolumn{1}{c|}{0000011} & LHU \\
\cline{2-11}
  

&
\multicolumn{4}{|c|}{imm[11:5]} &
\multicolumn{2}{c|}{rs2} &
\multicolumn{1}{c|}{rs1} &
\multicolumn{1}{c|}{000} &
\multicolumn{1}{c|}{imm[4:0]} &
\multicolumn{1}{c|}{0100011} & SB \\
\cline{2-11}
  

&
\multicolumn{4}{|c|}{imm[11:5]} &
\multicolumn{2}{c|}{rs2} &
\multicolumn{1}{c|}{rs1} &
\multicolumn{1}{c|}{001} &
\multicolumn{1}{c|}{imm[4:0]} &
\multicolumn{1}{c|}{0100011} & SH \\
\cline{2-11}
  

&
\multicolumn{4}{|c|}{imm[11:5]} &
\multicolumn{2}{c|}{rs2} &
\multicolumn{1}{c|}{rs1} &
\multicolumn{1}{c|}{010} &
\multicolumn{1}{c|}{imm[4:0]} &
\multicolumn{1}{c|}{0100011} & SW \\
\cline{2-11}
  

&
\multicolumn{6}{|c|}{imm[11:0]} &
\multicolumn{1}{c|}{rs1} &
\multicolumn{1}{c|}{000} &
\multicolumn{1}{c|}{rd} &
\multicolumn{1}{c|}{0010011} & ADDI \\
\cline{2-11}
  

&
\multicolumn{6}{|c|}{imm[11:0]} &
\multicolumn{1}{c|}{rs1} &
\multicolumn{1}{c|}{010} &
\multicolumn{1}{c|}{rd} &
\multicolumn{1}{c|}{0010011} & SLTI \\
\cline{2-11}
  

&
\multicolumn{6}{|c|}{imm[11:0]} &
\multicolumn{1}{c|}{rs1} &
\multicolumn{1}{c|}{011} &
\multicolumn{1}{c|}{rd} &
\multicolumn{1}{c|}{0010011} & SLTIU \\
\cline{2-11}
  

&
\multicolumn{6}{|c|}{imm[11:0]} &
\multicolumn{1}{c|}{rs1} &
\multicolumn{1}{c|}{100} &
\multicolumn{1}{c|}{rd} &
\multicolumn{1}{c|}{0010011} & XORI \\
\cline{2-11}
  

&
\multicolumn{6}{|c|}{imm[11:0]} &
\multicolumn{1}{c|}{rs1} &
\multicolumn{1}{c|}{110} &
\multicolumn{1}{c|}{rd} &
\multicolumn{1}{c|}{0010011} & ORI \\
\cline{2-11}
  

&
\multicolumn{6}{|c|}{imm[11:0]} &
\multicolumn{1}{c|}{rs1} &
\multicolumn{1}{c|}{111} &
\multicolumn{1}{c|}{rd} &
\multicolumn{1}{c|}{0010011} & ANDI \\
\cline{2-11}
  

&
\multicolumn{4}{|c|}{0000000} &
\multicolumn{2}{c|}{shamt} &
\multicolumn{1}{c|}{rs1} &
\multicolumn{1}{c|}{001} &
\multicolumn{1}{c|}{rd} &
\multicolumn{1}{c|}{0010011} & SLLI \\
\cline{2-11}
  

&
\multicolumn{4}{|c|}{0000000} &
\multicolumn{2}{c|}{shamt} &
\multicolumn{1}{c|}{rs1} &
\multicolumn{1}{c|}{101} &
\multicolumn{1}{c|}{rd} &
\multicolumn{1}{c|}{0010011} & SRLI \\
\cline{2-11}
  

&
\multicolumn{4}{|c|}{0100000} &
\multicolumn{2}{c|}{shamt} &
\multicolumn{1}{c|}{rs1} &
\multicolumn{1}{c|}{101} &
\multicolumn{1}{c|}{rd} &
\multicolumn{1}{c|}{0010011} & SRAI \\
\cline{2-11}
  

&
\multicolumn{4}{|c|}{0000000} &
\multicolumn{2}{c|}{rs2} &
\multicolumn{1}{c|}{rs1} &
\multicolumn{1}{c|}{000} &
\multicolumn{1}{c|}{rd} &
\multicolumn{1}{c|}{0110011} & ADD \\
\cline{2-11}
  

&
\multicolumn{4}{|c|}{0100000} &
\multicolumn{2}{c|}{rs2} &
\multicolumn{1}{c|}{rs1} &
\multicolumn{1}{c|}{000} &
\multicolumn{1}{c|}{rd} &
\multicolumn{1}{c|}{0110011} & SUB \\
\cline{2-11}
  

&
\multicolumn{4}{|c|}{0000000} &
\multicolumn{2}{c|}{rs2} &
\multicolumn{1}{c|}{rs1} &
\multicolumn{1}{c|}{001} &
\multicolumn{1}{c|}{rd} &
\multicolumn{1}{c|}{0110011} & SLL \\
\cline{2-11}
  

&
\multicolumn{4}{|c|}{0000000} &
\multicolumn{2}{c|}{rs2} &
\multicolumn{1}{c|}{rs1} &
\multicolumn{1}{c|}{010} &
\multicolumn{1}{c|}{rd} &
\multicolumn{1}{c|}{0110011} & SLT \\
\cline{2-11}
  

&
\multicolumn{4}{|c|}{0000000} &
\multicolumn{2}{c|}{rs2} &
\multicolumn{1}{c|}{rs1} &
\multicolumn{1}{c|}{011} &
\multicolumn{1}{c|}{rd} &
\multicolumn{1}{c|}{0110011} & SLTU \\
\cline{2-11}
  

&
\multicolumn{4}{|c|}{0000000} &
\multicolumn{2}{c|}{rs2} &
\multicolumn{1}{c|}{rs1} &
\multicolumn{1}{c|}{100} &
\multicolumn{1}{c|}{rd} &
\multicolumn{1}{c|}{0110011} & XOR \\
\cline{2-11}
  

&
\multicolumn{4}{|c|}{0000000} &
\multicolumn{2}{c|}{rs2} &
\multicolumn{1}{c|}{rs1} &
\multicolumn{1}{c|}{101} &
\multicolumn{1}{c|}{rd} &
\multicolumn{1}{c|}{0110011} & SRL \\
\cline{2-11}
  

&
\multicolumn{4}{|c|}{0100000} &
\multicolumn{2}{c|}{rs2} &
\multicolumn{1}{c|}{rs1} &
\multicolumn{1}{c|}{101} &
\multicolumn{1}{c|}{rd} &
\multicolumn{1}{c|}{0110011} & SRA \\
\cline{2-11}
  

&
\multicolumn{4}{|c|}{0000000} &
\multicolumn{2}{c|}{rs2} &
\multicolumn{1}{c|}{rs1} &
\multicolumn{1}{c|}{110} &
\multicolumn{1}{c|}{rd} &
\multicolumn{1}{c|}{0110011} & OR \\
\cline{2-11}
  

&
\multicolumn{4}{|c|}{0000000} &
\multicolumn{2}{c|}{rs2} &
\multicolumn{1}{c|}{rs1} &
\multicolumn{1}{c|}{111} &
\multicolumn{1}{c|}{rd} &
\multicolumn{1}{c|}{0110011} & AND \\
\cline{2-11}
  

&
\multicolumn{2}{|c|}{fm} &
\multicolumn{3}{c|}{pred} &
\multicolumn{1}{c|}{succ} &
\multicolumn{1}{c|}{rs1} &
\multicolumn{1}{c|}{000} &
\multicolumn{1}{c|}{rd} &
\multicolumn{1}{c|}{0001111} & FENCE \\
\cline{2-11}
  

&
\multicolumn{2}{|c|}{1000} &
\multicolumn{3}{c|}{0011} &
\multicolumn{1}{c|}{0011} &
\multicolumn{1}{c|}{00000} &
\multicolumn{1}{c|}{000} &
\multicolumn{1}{c|}{00000} &
\multicolumn{1}{c|}{0001111} & FENCE.TSO \\
\cline{2-11}
  

&
\multicolumn{2}{|c|}{0000} &
\multicolumn{3}{c|}{0001} &
\multicolumn{1}{c|}{0000} &
\multicolumn{1}{c|}{00000} &
\multicolumn{1}{c|}{000} &
\multicolumn{1}{c|}{00000} &
\multicolumn{1}{c|}{0001111} & PAUSE \\
\cline{2-11}
  

&
\multicolumn{6}{|c|}{000000000000} &
\multicolumn{1}{c|}{00000} &
\multicolumn{1}{c|}{000} &
\multicolumn{1}{c|}{00000} &
\multicolumn{1}{c|}{1110011} & ECALL \\
\cline{2-11}
  

&
\multicolumn{6}{|c|}{000000000001} &
\multicolumn{1}{c|}{00000} &
\multicolumn{1}{c|}{000} &
\multicolumn{1}{c|}{00000} &
\multicolumn{1}{c|}{1110011} & EBREAK \\
\cline{2-11}
  

\end{tabular}
\end{center}
\end{small}

\end{table}
  

\newpage

\begin{table}[p]
\begin{small}
\begin{center}
\begin{tabular}{p{0in}p{0.4in}p{0.05in}p{0.05in}p{0.05in}p{0.05in}p{0.4in}p{0.6in}p{0.4in}p{0.6in}p{0.7in}l}
& & & & & & & & & & \\
                      &
\multicolumn{1}{l}{\instbit{31}} &
\multicolumn{1}{r}{\instbit{27}} &
\instbit{26} &
\instbit{25} &
\multicolumn{1}{l}{\instbit{24}} &
\multicolumn{1}{r}{\instbit{20}} &
\instbitrange{19}{15} &
\instbitrange{14}{12} &
\instbitrange{11}{7} &
\instbitrange{6}{0} \\
\cline{2-11}


&
\multicolumn{4}{|c|}{funct7} &
\multicolumn{2}{c|}{rs2} &
\multicolumn{1}{c|}{rs1} &
\multicolumn{1}{c|}{funct3} &
\multicolumn{1}{c|}{rd} &
\multicolumn{1}{c|}{opcode} & R-type \\
\cline{2-11}


&
\multicolumn{6}{|c|}{imm[11:0]} &
\multicolumn{1}{c|}{rs1} &
\multicolumn{1}{c|}{funct3} &
\multicolumn{1}{c|}{rd} &
\multicolumn{1}{c|}{opcode} & I-type \\
\cline{2-11}


&
\multicolumn{4}{|c|}{imm[11:5]} &
\multicolumn{2}{c|}{rs2} &
\multicolumn{1}{c|}{rs1} &
\multicolumn{1}{c|}{funct3} &
\multicolumn{1}{c|}{imm[4:0]} &
\multicolumn{1}{c|}{opcode} & S-type \\
\cline{2-11}


&
\multicolumn{10}{c}{} & \\
&
\multicolumn{10}{c}{\bf RV64I Base Instruction Set (in addition to RV32I)} & \\
\cline{2-11}
  

&
\multicolumn{6}{|c|}{imm[11:0]} &
\multicolumn{1}{c|}{rs1} &
\multicolumn{1}{c|}{110} &
\multicolumn{1}{c|}{rd} &
\multicolumn{1}{c|}{0000011} & LWU \\
\cline{2-11}
  

&
\multicolumn{6}{|c|}{imm[11:0]} &
\multicolumn{1}{c|}{rs1} &
\multicolumn{1}{c|}{011} &
\multicolumn{1}{c|}{rd} &
\multicolumn{1}{c|}{0000011} & LD \\
\cline{2-11}
  

&
\multicolumn{4}{|c|}{imm[11:5]} &
\multicolumn{2}{c|}{rs2} &
\multicolumn{1}{c|}{rs1} &
\multicolumn{1}{c|}{011} &
\multicolumn{1}{c|}{imm[4:0]} &
\multicolumn{1}{c|}{0100011} & SD \\
\cline{2-11}
  

&
\multicolumn{3}{|c|}{000000} &
\multicolumn{3}{c|}{shamt} &
\multicolumn{1}{c|}{rs1} &
\multicolumn{1}{c|}{001} &
\multicolumn{1}{c|}{rd} &
\multicolumn{1}{c|}{0010011} & SLLI \\
\cline{2-11}
  

&
\multicolumn{3}{|c|}{000000} &
\multicolumn{3}{c|}{shamt} &
\multicolumn{1}{c|}{rs1} &
\multicolumn{1}{c|}{101} &
\multicolumn{1}{c|}{rd} &
\multicolumn{1}{c|}{0010011} & SRLI \\
\cline{2-11}
  

&
\multicolumn{3}{|c|}{010000} &
\multicolumn{3}{c|}{shamt} &
\multicolumn{1}{c|}{rs1} &
\multicolumn{1}{c|}{101} &
\multicolumn{1}{c|}{rd} &
\multicolumn{1}{c|}{0010011} & SRAI \\
\cline{2-11}
  

&
\multicolumn{6}{|c|}{imm[11:0]} &
\multicolumn{1}{c|}{rs1} &
\multicolumn{1}{c|}{000} &
\multicolumn{1}{c|}{rd} &
\multicolumn{1}{c|}{0011011} & ADDIW \\
\cline{2-11}
  

&
\multicolumn{4}{|c|}{0000000} &
\multicolumn{2}{c|}{shamt} &
\multicolumn{1}{c|}{rs1} &
\multicolumn{1}{c|}{001} &
\multicolumn{1}{c|}{rd} &
\multicolumn{1}{c|}{0011011} & SLLIW \\
\cline{2-11}
  

&
\multicolumn{4}{|c|}{0000000} &
\multicolumn{2}{c|}{shamt} &
\multicolumn{1}{c|}{rs1} &
\multicolumn{1}{c|}{101} &
\multicolumn{1}{c|}{rd} &
\multicolumn{1}{c|}{0011011} & SRLIW \\
\cline{2-11}
  

&
\multicolumn{4}{|c|}{0100000} &
\multicolumn{2}{c|}{shamt} &
\multicolumn{1}{c|}{rs1} &
\multicolumn{1}{c|}{101} &
\multicolumn{1}{c|}{rd} &
\multicolumn{1}{c|}{0011011} & SRAIW \\
\cline{2-11}
  

&
\multicolumn{4}{|c|}{0000000} &
\multicolumn{2}{c|}{rs2} &
\multicolumn{1}{c|}{rs1} &
\multicolumn{1}{c|}{000} &
\multicolumn{1}{c|}{rd} &
\multicolumn{1}{c|}{0111011} & ADDW \\
\cline{2-11}
  

&
\multicolumn{4}{|c|}{0100000} &
\multicolumn{2}{c|}{rs2} &
\multicolumn{1}{c|}{rs1} &
\multicolumn{1}{c|}{000} &
\multicolumn{1}{c|}{rd} &
\multicolumn{1}{c|}{0111011} & SUBW \\
\cline{2-11}
  

&
\multicolumn{4}{|c|}{0000000} &
\multicolumn{2}{c|}{rs2} &
\multicolumn{1}{c|}{rs1} &
\multicolumn{1}{c|}{001} &
\multicolumn{1}{c|}{rd} &
\multicolumn{1}{c|}{0111011} & SLLW \\
\cline{2-11}
  

&
\multicolumn{4}{|c|}{0000000} &
\multicolumn{2}{c|}{rs2} &
\multicolumn{1}{c|}{rs1} &
\multicolumn{1}{c|}{101} &
\multicolumn{1}{c|}{rd} &
\multicolumn{1}{c|}{0111011} & SRLW \\
\cline{2-11}
  

&
\multicolumn{4}{|c|}{0100000} &
\multicolumn{2}{c|}{rs2} &
\multicolumn{1}{c|}{rs1} &
\multicolumn{1}{c|}{101} &
\multicolumn{1}{c|}{rd} &
\multicolumn{1}{c|}{0111011} & SRAW \\
\cline{2-11}
  

&
\multicolumn{10}{c}{} & \\
&
\multicolumn{10}{c}{\bf RV32/RV64 \emph{Zifencei} Standard Extension} & \\
\cline{2-11}
  

&
\multicolumn{6}{|c|}{imm[11:0]} &
\multicolumn{1}{c|}{rs1} &
\multicolumn{1}{c|}{001} &
\multicolumn{1}{c|}{rd} &
\multicolumn{1}{c|}{0001111} & FENCE.I \\
\cline{2-11}
  

&
\multicolumn{10}{c}{} & \\
&
\multicolumn{10}{c}{\bf RV32/RV64 \emph{Zicsr} Standard Extension} & \\
\cline{2-11}
  

&
\multicolumn{6}{|c|}{csr} &
\multicolumn{1}{c|}{rs1} &
\multicolumn{1}{c|}{001} &
\multicolumn{1}{c|}{rd} &
\multicolumn{1}{c|}{1110011} & CSRRW \\
\cline{2-11}
  

&
\multicolumn{6}{|c|}{csr} &
\multicolumn{1}{c|}{rs1} &
\multicolumn{1}{c|}{010} &
\multicolumn{1}{c|}{rd} &
\multicolumn{1}{c|}{1110011} & CSRRS \\
\cline{2-11}
  

&
\multicolumn{6}{|c|}{csr} &
\multicolumn{1}{c|}{rs1} &
\multicolumn{1}{c|}{011} &
\multicolumn{1}{c|}{rd} &
\multicolumn{1}{c|}{1110011} & CSRRC \\
\cline{2-11}
  

&
\multicolumn{6}{|c|}{csr} &
\multicolumn{1}{c|}{uimm} &
\multicolumn{1}{c|}{101} &
\multicolumn{1}{c|}{rd} &
\multicolumn{1}{c|}{1110011} & CSRRWI \\
\cline{2-11}
  

&
\multicolumn{6}{|c|}{csr} &
\multicolumn{1}{c|}{uimm} &
\multicolumn{1}{c|}{110} &
\multicolumn{1}{c|}{rd} &
\multicolumn{1}{c|}{1110011} & CSRRSI \\
\cline{2-11}
  

&
\multicolumn{6}{|c|}{csr} &
\multicolumn{1}{c|}{uimm} &
\multicolumn{1}{c|}{111} &
\multicolumn{1}{c|}{rd} &
\multicolumn{1}{c|}{1110011} & CSRRCI \\
\cline{2-11}
  

&
\multicolumn{10}{c}{} & \\
&
\multicolumn{10}{c}{\bf RV32M Standard Extension} & \\
\cline{2-11}
  

&
\multicolumn{4}{|c|}{0000001} &
\multicolumn{2}{c|}{rs2} &
\multicolumn{1}{c|}{rs1} &
\multicolumn{1}{c|}{000} &
\multicolumn{1}{c|}{rd} &
\multicolumn{1}{c|}{0110011} & MUL \\
\cline{2-11}
  

&
\multicolumn{4}{|c|}{0000001} &
\multicolumn{2}{c|}{rs2} &
\multicolumn{1}{c|}{rs1} &
\multicolumn{1}{c|}{001} &
\multicolumn{1}{c|}{rd} &
\multicolumn{1}{c|}{0110011} & MULH \\
\cline{2-11}
  

&
\multicolumn{4}{|c|}{0000001} &
\multicolumn{2}{c|}{rs2} &
\multicolumn{1}{c|}{rs1} &
\multicolumn{1}{c|}{010} &
\multicolumn{1}{c|}{rd} &
\multicolumn{1}{c|}{0110011} & MULHSU \\
\cline{2-11}
  

&
\multicolumn{4}{|c|}{0000001} &
\multicolumn{2}{c|}{rs2} &
\multicolumn{1}{c|}{rs1} &
\multicolumn{1}{c|}{011} &
\multicolumn{1}{c|}{rd} &
\multicolumn{1}{c|}{0110011} & MULHU \\
\cline{2-11}
  

&
\multicolumn{4}{|c|}{0000001} &
\multicolumn{2}{c|}{rs2} &
\multicolumn{1}{c|}{rs1} &
\multicolumn{1}{c|}{100} &
\multicolumn{1}{c|}{rd} &
\multicolumn{1}{c|}{0110011} & DIV \\
\cline{2-11}
  

&
\multicolumn{4}{|c|}{0000001} &
\multicolumn{2}{c|}{rs2} &
\multicolumn{1}{c|}{rs1} &
\multicolumn{1}{c|}{101} &
\multicolumn{1}{c|}{rd} &
\multicolumn{1}{c|}{0110011} & DIVU \\
\cline{2-11}
  

&
\multicolumn{4}{|c|}{0000001} &
\multicolumn{2}{c|}{rs2} &
\multicolumn{1}{c|}{rs1} &
\multicolumn{1}{c|}{110} &
\multicolumn{1}{c|}{rd} &
\multicolumn{1}{c|}{0110011} & REM \\
\cline{2-11}
  

&
\multicolumn{4}{|c|}{0000001} &
\multicolumn{2}{c|}{rs2} &
\multicolumn{1}{c|}{rs1} &
\multicolumn{1}{c|}{111} &
\multicolumn{1}{c|}{rd} &
\multicolumn{1}{c|}{0110011} & REMU \\
\cline{2-11}
  

&
\multicolumn{10}{c}{} & \\
&
\multicolumn{10}{c}{\bf RV64M Standard Extension (in addition to RV32M)} & \\
\cline{2-11}
  

&
\multicolumn{4}{|c|}{0000001} &
\multicolumn{2}{c|}{rs2} &
\multicolumn{1}{c|}{rs1} &
\multicolumn{1}{c|}{000} &
\multicolumn{1}{c|}{rd} &
\multicolumn{1}{c|}{0111011} & MULW \\
\cline{2-11}
  

&
\multicolumn{4}{|c|}{0000001} &
\multicolumn{2}{c|}{rs2} &
\multicolumn{1}{c|}{rs1} &
\multicolumn{1}{c|}{100} &
\multicolumn{1}{c|}{rd} &
\multicolumn{1}{c|}{0111011} & DIVW \\
\cline{2-11}
  

&
\multicolumn{4}{|c|}{0000001} &
\multicolumn{2}{c|}{rs2} &
\multicolumn{1}{c|}{rs1} &
\multicolumn{1}{c|}{101} &
\multicolumn{1}{c|}{rd} &
\multicolumn{1}{c|}{0111011} & DIVUW \\
\cline{2-11}
  

&
\multicolumn{4}{|c|}{0000001} &
\multicolumn{2}{c|}{rs2} &
\multicolumn{1}{c|}{rs1} &
\multicolumn{1}{c|}{110} &
\multicolumn{1}{c|}{rd} &
\multicolumn{1}{c|}{0111011} & REMW \\
\cline{2-11}
  

&
\multicolumn{4}{|c|}{0000001} &
\multicolumn{2}{c|}{rs2} &
\multicolumn{1}{c|}{rs1} &
\multicolumn{1}{c|}{111} &
\multicolumn{1}{c|}{rd} &
\multicolumn{1}{c|}{0111011} & REMUW \\
\cline{2-11}
  

\end{tabular}
\end{center}
\end{small}

\end{table}
  

\newpage

\begin{table}[p]
\begin{small}
\begin{center}
\begin{tabular}{p{0in}p{0.4in}p{0.05in}p{0.05in}p{0.05in}p{0.05in}p{0.4in}p{0.6in}p{0.4in}p{0.6in}p{0.7in}l}
& & & & & & & & & & \\
                      &
\multicolumn{1}{l}{\instbit{31}} &
\multicolumn{1}{r}{\instbit{27}} &
\instbit{26} &
\instbit{25} &
\multicolumn{1}{l}{\instbit{24}} &
\multicolumn{1}{r}{\instbit{20}} &
\instbitrange{19}{15} &
\instbitrange{14}{12} &
\instbitrange{11}{7} &
\instbitrange{6}{0} \\
\cline{2-11}


&
\multicolumn{4}{|c|}{funct7} &
\multicolumn{2}{c|}{rs2} &
\multicolumn{1}{c|}{rs1} &
\multicolumn{1}{c|}{funct3} &
\multicolumn{1}{c|}{rd} &
\multicolumn{1}{c|}{opcode} & R-type \\
\cline{2-11}


&
\multicolumn{10}{c}{} & \\
&
\multicolumn{10}{c}{\bf RV32A Standard Extension} & \\
\cline{2-11}
  

&
\multicolumn{2}{|c|}{00010} &
\multicolumn{1}{c|}{aq} &
\multicolumn{1}{c|}{rl} &
\multicolumn{2}{c|}{00000} &
\multicolumn{1}{c|}{rs1} &
\multicolumn{1}{c|}{010} &
\multicolumn{1}{c|}{rd} &
\multicolumn{1}{c|}{0101111} & LR.W \\
\cline{2-11}
  

&
\multicolumn{2}{|c|}{00011} &
\multicolumn{1}{c|}{aq} &
\multicolumn{1}{c|}{rl} &
\multicolumn{2}{c|}{rs2} &
\multicolumn{1}{c|}{rs1} &
\multicolumn{1}{c|}{010} &
\multicolumn{1}{c|}{rd} &
\multicolumn{1}{c|}{0101111} & SC.W \\
\cline{2-11}
  

&
\multicolumn{2}{|c|}{00001} &
\multicolumn{1}{c|}{aq} &
\multicolumn{1}{c|}{rl} &
\multicolumn{2}{c|}{rs2} &
\multicolumn{1}{c|}{rs1} &
\multicolumn{1}{c|}{010} &
\multicolumn{1}{c|}{rd} &
\multicolumn{1}{c|}{0101111} & AMOSWAP.W \\
\cline{2-11}
  

&
\multicolumn{2}{|c|}{00000} &
\multicolumn{1}{c|}{aq} &
\multicolumn{1}{c|}{rl} &
\multicolumn{2}{c|}{rs2} &
\multicolumn{1}{c|}{rs1} &
\multicolumn{1}{c|}{010} &
\multicolumn{1}{c|}{rd} &
\multicolumn{1}{c|}{0101111} & AMOADD.W \\
\cline{2-11}
  

&
\multicolumn{2}{|c|}{00100} &
\multicolumn{1}{c|}{aq} &
\multicolumn{1}{c|}{rl} &
\multicolumn{2}{c|}{rs2} &
\multicolumn{1}{c|}{rs1} &
\multicolumn{1}{c|}{010} &
\multicolumn{1}{c|}{rd} &
\multicolumn{1}{c|}{0101111} & AMOXOR.W \\
\cline{2-11}
  

&
\multicolumn{2}{|c|}{01100} &
\multicolumn{1}{c|}{aq} &
\multicolumn{1}{c|}{rl} &
\multicolumn{2}{c|}{rs2} &
\multicolumn{1}{c|}{rs1} &
\multicolumn{1}{c|}{010} &
\multicolumn{1}{c|}{rd} &
\multicolumn{1}{c|}{0101111} & AMOAND.W \\
\cline{2-11}
  

&
\multicolumn{2}{|c|}{01000} &
\multicolumn{1}{c|}{aq} &
\multicolumn{1}{c|}{rl} &
\multicolumn{2}{c|}{rs2} &
\multicolumn{1}{c|}{rs1} &
\multicolumn{1}{c|}{010} &
\multicolumn{1}{c|}{rd} &
\multicolumn{1}{c|}{0101111} & AMOOR.W \\
\cline{2-11}
  

&
\multicolumn{2}{|c|}{10000} &
\multicolumn{1}{c|}{aq} &
\multicolumn{1}{c|}{rl} &
\multicolumn{2}{c|}{rs2} &
\multicolumn{1}{c|}{rs1} &
\multicolumn{1}{c|}{010} &
\multicolumn{1}{c|}{rd} &
\multicolumn{1}{c|}{0101111} & AMOMIN.W \\
\cline{2-11}
  

&
\multicolumn{2}{|c|}{10100} &
\multicolumn{1}{c|}{aq} &
\multicolumn{1}{c|}{rl} &
\multicolumn{2}{c|}{rs2} &
\multicolumn{1}{c|}{rs1} &
\multicolumn{1}{c|}{010} &
\multicolumn{1}{c|}{rd} &
\multicolumn{1}{c|}{0101111} & AMOMAX.W \\
\cline{2-11}
  

&
\multicolumn{2}{|c|}{11000} &
\multicolumn{1}{c|}{aq} &
\multicolumn{1}{c|}{rl} &
\multicolumn{2}{c|}{rs2} &
\multicolumn{1}{c|}{rs1} &
\multicolumn{1}{c|}{010} &
\multicolumn{1}{c|}{rd} &
\multicolumn{1}{c|}{0101111} & AMOMINU.W \\
\cline{2-11}
  

&
\multicolumn{2}{|c|}{11100} &
\multicolumn{1}{c|}{aq} &
\multicolumn{1}{c|}{rl} &
\multicolumn{2}{c|}{rs2} &
\multicolumn{1}{c|}{rs1} &
\multicolumn{1}{c|}{010} &
\multicolumn{1}{c|}{rd} &
\multicolumn{1}{c|}{0101111} & AMOMAXU.W \\
\cline{2-11}
  

&
\multicolumn{10}{c}{} & \\
&
\multicolumn{10}{c}{\bf RV64A Standard Extension (in addition to RV32A)} & \\
\cline{2-11}
  

&
\multicolumn{2}{|c|}{00010} &
\multicolumn{1}{c|}{aq} &
\multicolumn{1}{c|}{rl} &
\multicolumn{2}{c|}{00000} &
\multicolumn{1}{c|}{rs1} &
\multicolumn{1}{c|}{011} &
\multicolumn{1}{c|}{rd} &
\multicolumn{1}{c|}{0101111} & LR.D \\
\cline{2-11}
  

&
\multicolumn{2}{|c|}{00011} &
\multicolumn{1}{c|}{aq} &
\multicolumn{1}{c|}{rl} &
\multicolumn{2}{c|}{rs2} &
\multicolumn{1}{c|}{rs1} &
\multicolumn{1}{c|}{011} &
\multicolumn{1}{c|}{rd} &
\multicolumn{1}{c|}{0101111} & SC.D \\
\cline{2-11}
  

&
\multicolumn{2}{|c|}{00001} &
\multicolumn{1}{c|}{aq} &
\multicolumn{1}{c|}{rl} &
\multicolumn{2}{c|}{rs2} &
\multicolumn{1}{c|}{rs1} &
\multicolumn{1}{c|}{011} &
\multicolumn{1}{c|}{rd} &
\multicolumn{1}{c|}{0101111} & AMOSWAP.D \\
\cline{2-11}
  

&
\multicolumn{2}{|c|}{00000} &
\multicolumn{1}{c|}{aq} &
\multicolumn{1}{c|}{rl} &
\multicolumn{2}{c|}{rs2} &
\multicolumn{1}{c|}{rs1} &
\multicolumn{1}{c|}{011} &
\multicolumn{1}{c|}{rd} &
\multicolumn{1}{c|}{0101111} & AMOADD.D \\
\cline{2-11}
  

&
\multicolumn{2}{|c|}{00100} &
\multicolumn{1}{c|}{aq} &
\multicolumn{1}{c|}{rl} &
\multicolumn{2}{c|}{rs2} &
\multicolumn{1}{c|}{rs1} &
\multicolumn{1}{c|}{011} &
\multicolumn{1}{c|}{rd} &
\multicolumn{1}{c|}{0101111} & AMOXOR.D \\
\cline{2-11}
  

&
\multicolumn{2}{|c|}{01100} &
\multicolumn{1}{c|}{aq} &
\multicolumn{1}{c|}{rl} &
\multicolumn{2}{c|}{rs2} &
\multicolumn{1}{c|}{rs1} &
\multicolumn{1}{c|}{011} &
\multicolumn{1}{c|}{rd} &
\multicolumn{1}{c|}{0101111} & AMOAND.D \\
\cline{2-11}
  

&
\multicolumn{2}{|c|}{01000} &
\multicolumn{1}{c|}{aq} &
\multicolumn{1}{c|}{rl} &
\multicolumn{2}{c|}{rs2} &
\multicolumn{1}{c|}{rs1} &
\multicolumn{1}{c|}{011} &
\multicolumn{1}{c|}{rd} &
\multicolumn{1}{c|}{0101111} & AMOOR.D \\
\cline{2-11}
  

&
\multicolumn{2}{|c|}{10000} &
\multicolumn{1}{c|}{aq} &
\multicolumn{1}{c|}{rl} &
\multicolumn{2}{c|}{rs2} &
\multicolumn{1}{c|}{rs1} &
\multicolumn{1}{c|}{011} &
\multicolumn{1}{c|}{rd} &
\multicolumn{1}{c|}{0101111} & AMOMIN.D \\
\cline{2-11}
  

&
\multicolumn{2}{|c|}{10100} &
\multicolumn{1}{c|}{aq} &
\multicolumn{1}{c|}{rl} &
\multicolumn{2}{c|}{rs2} &
\multicolumn{1}{c|}{rs1} &
\multicolumn{1}{c|}{011} &
\multicolumn{1}{c|}{rd} &
\multicolumn{1}{c|}{0101111} & AMOMAX.D \\
\cline{2-11}
  

&
\multicolumn{2}{|c|}{11000} &
\multicolumn{1}{c|}{aq} &
\multicolumn{1}{c|}{rl} &
\multicolumn{2}{c|}{rs2} &
\multicolumn{1}{c|}{rs1} &
\multicolumn{1}{c|}{011} &
\multicolumn{1}{c|}{rd} &
\multicolumn{1}{c|}{0101111} & AMOMINU.D \\
\cline{2-11}
  

&
\multicolumn{2}{|c|}{11100} &
\multicolumn{1}{c|}{aq} &
\multicolumn{1}{c|}{rl} &
\multicolumn{2}{c|}{rs2} &
\multicolumn{1}{c|}{rs1} &
\multicolumn{1}{c|}{011} &
\multicolumn{1}{c|}{rd} &
\multicolumn{1}{c|}{0101111} & AMOMAXU.D \\
\cline{2-11}
  

\end{tabular}
\end{center}
\end{small}

\end{table}
  

\newpage

\begin{table}[p]
\begin{small}
\begin{center}
\begin{tabular}{p{0in}p{0.4in}p{0.05in}p{0.05in}p{0.05in}p{0.05in}p{0.4in}p{0.6in}p{0.4in}p{0.6in}p{0.7in}l}
& & & & & & & & & & \\
                      &
\multicolumn{1}{l}{\instbit{31}} &
\multicolumn{1}{r}{\instbit{27}} &
\instbit{26} &
\instbit{25} &
\multicolumn{1}{l}{\instbit{24}} &
\multicolumn{1}{r}{\instbit{20}} &
\instbitrange{19}{15} &
\instbitrange{14}{12} &
\instbitrange{11}{7} &
\instbitrange{6}{0} \\
\cline{2-11}


&
\multicolumn{4}{|c|}{funct7} &
\multicolumn{2}{c|}{rs2} &
\multicolumn{1}{c|}{rs1} &
\multicolumn{1}{c|}{funct3} &
\multicolumn{1}{c|}{rd} &
\multicolumn{1}{c|}{opcode} & R-type \\
\cline{2-11}


&
\multicolumn{2}{|c|}{rs3} &
\multicolumn{2}{c|}{funct2} &
\multicolumn{2}{c|}{rs2} &
\multicolumn{1}{c|}{rs1} &
\multicolumn{1}{c|}{funct3} &
\multicolumn{1}{c|}{rd} &
\multicolumn{1}{c|}{opcode} & R4-type \\
\cline{2-11}
  

&
\multicolumn{6}{|c|}{imm[11:0]} &
\multicolumn{1}{c|}{rs1} &
\multicolumn{1}{c|}{funct3} &
\multicolumn{1}{c|}{rd} &
\multicolumn{1}{c|}{opcode} & I-type \\
\cline{2-11}


&
\multicolumn{4}{|c|}{imm[11:5]} &
\multicolumn{2}{c|}{rs2} &
\multicolumn{1}{c|}{rs1} &
\multicolumn{1}{c|}{funct3} &
\multicolumn{1}{c|}{imm[4:0]} &
\multicolumn{1}{c|}{opcode} & S-type \\
\cline{2-11}


&
\multicolumn{10}{c}{} & \\
&
\multicolumn{10}{c}{\bf RV32F Standard Extension} & \\
\cline{2-11}
  

&
\multicolumn{6}{|c|}{imm[11:0]} &
\multicolumn{1}{c|}{rs1} &
\multicolumn{1}{c|}{010} &
\multicolumn{1}{c|}{rd} &
\multicolumn{1}{c|}{0000111} & FLW \\
\cline{2-11}
  

&
\multicolumn{4}{|c|}{imm[11:5]} &
\multicolumn{2}{c|}{rs2} &
\multicolumn{1}{c|}{rs1} &
\multicolumn{1}{c|}{010} &
\multicolumn{1}{c|}{imm[4:0]} &
\multicolumn{1}{c|}{0100111} & FSW \\
\cline{2-11}
  

&
\multicolumn{2}{|c|}{rs3} &
\multicolumn{2}{c|}{00} &
\multicolumn{2}{c|}{rs2} &
\multicolumn{1}{c|}{rs1} &
\multicolumn{1}{c|}{rm} &
\multicolumn{1}{c|}{rd} &
\multicolumn{1}{c|}{1000011} & FMADD.S \\
\cline{2-11}
  

&
\multicolumn{2}{|c|}{rs3} &
\multicolumn{2}{c|}{00} &
\multicolumn{2}{c|}{rs2} &
\multicolumn{1}{c|}{rs1} &
\multicolumn{1}{c|}{rm} &
\multicolumn{1}{c|}{rd} &
\multicolumn{1}{c|}{1000111} & FMSUB.S \\
\cline{2-11}
  

&
\multicolumn{2}{|c|}{rs3} &
\multicolumn{2}{c|}{00} &
\multicolumn{2}{c|}{rs2} &
\multicolumn{1}{c|}{rs1} &
\multicolumn{1}{c|}{rm} &
\multicolumn{1}{c|}{rd} &
\multicolumn{1}{c|}{1001011} & FNMSUB.S \\
\cline{2-11}
  

&
\multicolumn{2}{|c|}{rs3} &
\multicolumn{2}{c|}{00} &
\multicolumn{2}{c|}{rs2} &
\multicolumn{1}{c|}{rs1} &
\multicolumn{1}{c|}{rm} &
\multicolumn{1}{c|}{rd} &
\multicolumn{1}{c|}{1001111} & FNMADD.S \\
\cline{2-11}
  

&
\multicolumn{4}{|c|}{0000000} &
\multicolumn{2}{c|}{rs2} &
\multicolumn{1}{c|}{rs1} &
\multicolumn{1}{c|}{rm} &
\multicolumn{1}{c|}{rd} &
\multicolumn{1}{c|}{1010011} & FADD.S \\
\cline{2-11}
  

&
\multicolumn{4}{|c|}{0000100} &
\multicolumn{2}{c|}{rs2} &
\multicolumn{1}{c|}{rs1} &
\multicolumn{1}{c|}{rm} &
\multicolumn{1}{c|}{rd} &
\multicolumn{1}{c|}{1010011} & FSUB.S \\
\cline{2-11}
  

&
\multicolumn{4}{|c|}{0001000} &
\multicolumn{2}{c|}{rs2} &
\multicolumn{1}{c|}{rs1} &
\multicolumn{1}{c|}{rm} &
\multicolumn{1}{c|}{rd} &
\multicolumn{1}{c|}{1010011} & FMUL.S \\
\cline{2-11}
  

&
\multicolumn{4}{|c|}{0001100} &
\multicolumn{2}{c|}{rs2} &
\multicolumn{1}{c|}{rs1} &
\multicolumn{1}{c|}{rm} &
\multicolumn{1}{c|}{rd} &
\multicolumn{1}{c|}{1010011} & FDIV.S \\
\cline{2-11}
  

&
\multicolumn{4}{|c|}{0101100} &
\multicolumn{2}{c|}{00000} &
\multicolumn{1}{c|}{rs1} &
\multicolumn{1}{c|}{rm} &
\multicolumn{1}{c|}{rd} &
\multicolumn{1}{c|}{1010011} & FSQRT.S \\
\cline{2-11}
  

&
\multicolumn{4}{|c|}{0010000} &
\multicolumn{2}{c|}{rs2} &
\multicolumn{1}{c|}{rs1} &
\multicolumn{1}{c|}{000} &
\multicolumn{1}{c|}{rd} &
\multicolumn{1}{c|}{1010011} & FSGNJ.S \\
\cline{2-11}
  

&
\multicolumn{4}{|c|}{0010000} &
\multicolumn{2}{c|}{rs2} &
\multicolumn{1}{c|}{rs1} &
\multicolumn{1}{c|}{001} &
\multicolumn{1}{c|}{rd} &
\multicolumn{1}{c|}{1010011} & FSGNJN.S \\
\cline{2-11}
  

&
\multicolumn{4}{|c|}{0010000} &
\multicolumn{2}{c|}{rs2} &
\multicolumn{1}{c|}{rs1} &
\multicolumn{1}{c|}{010} &
\multicolumn{1}{c|}{rd} &
\multicolumn{1}{c|}{1010011} & FSGNJX.S \\
\cline{2-11}
  

&
\multicolumn{4}{|c|}{0010100} &
\multicolumn{2}{c|}{rs2} &
\multicolumn{1}{c|}{rs1} &
\multicolumn{1}{c|}{000} &
\multicolumn{1}{c|}{rd} &
\multicolumn{1}{c|}{1010011} & FMIN.S \\
\cline{2-11}
  

&
\multicolumn{4}{|c|}{0010100} &
\multicolumn{2}{c|}{rs2} &
\multicolumn{1}{c|}{rs1} &
\multicolumn{1}{c|}{001} &
\multicolumn{1}{c|}{rd} &
\multicolumn{1}{c|}{1010011} & FMAX.S \\
\cline{2-11}
  

&
\multicolumn{4}{|c|}{1100000} &
\multicolumn{2}{c|}{00000} &
\multicolumn{1}{c|}{rs1} &
\multicolumn{1}{c|}{rm} &
\multicolumn{1}{c|}{rd} &
\multicolumn{1}{c|}{1010011} & FCVT.W.S \\
\cline{2-11}
  

&
\multicolumn{4}{|c|}{1100000} &
\multicolumn{2}{c|}{00001} &
\multicolumn{1}{c|}{rs1} &
\multicolumn{1}{c|}{rm} &
\multicolumn{1}{c|}{rd} &
\multicolumn{1}{c|}{1010011} & FCVT.WU.S \\
\cline{2-11}
  

&
\multicolumn{4}{|c|}{1110000} &
\multicolumn{2}{c|}{00000} &
\multicolumn{1}{c|}{rs1} &
\multicolumn{1}{c|}{000} &
\multicolumn{1}{c|}{rd} &
\multicolumn{1}{c|}{1010011} & FMV.X.W \\
\cline{2-11}
  

&
\multicolumn{4}{|c|}{1010000} &
\multicolumn{2}{c|}{rs2} &
\multicolumn{1}{c|}{rs1} &
\multicolumn{1}{c|}{010} &
\multicolumn{1}{c|}{rd} &
\multicolumn{1}{c|}{1010011} & FEQ.S \\
\cline{2-11}
  

&
\multicolumn{4}{|c|}{1010000} &
\multicolumn{2}{c|}{rs2} &
\multicolumn{1}{c|}{rs1} &
\multicolumn{1}{c|}{001} &
\multicolumn{1}{c|}{rd} &
\multicolumn{1}{c|}{1010011} & FLT.S \\
\cline{2-11}
  

&
\multicolumn{4}{|c|}{1010000} &
\multicolumn{2}{c|}{rs2} &
\multicolumn{1}{c|}{rs1} &
\multicolumn{1}{c|}{000} &
\multicolumn{1}{c|}{rd} &
\multicolumn{1}{c|}{1010011} & FLE.S \\
\cline{2-11}
  

&
\multicolumn{4}{|c|}{1110000} &
\multicolumn{2}{c|}{00000} &
\multicolumn{1}{c|}{rs1} &
\multicolumn{1}{c|}{001} &
\multicolumn{1}{c|}{rd} &
\multicolumn{1}{c|}{1010011} & FCLASS.S \\
\cline{2-11}
  

&
\multicolumn{4}{|c|}{1101000} &
\multicolumn{2}{c|}{00000} &
\multicolumn{1}{c|}{rs1} &
\multicolumn{1}{c|}{rm} &
\multicolumn{1}{c|}{rd} &
\multicolumn{1}{c|}{1010011} & FCVT.S.W \\
\cline{2-11}
  

&
\multicolumn{4}{|c|}{1101000} &
\multicolumn{2}{c|}{00001} &
\multicolumn{1}{c|}{rs1} &
\multicolumn{1}{c|}{rm} &
\multicolumn{1}{c|}{rd} &
\multicolumn{1}{c|}{1010011} & FCVT.S.WU \\
\cline{2-11}
  

&
\multicolumn{4}{|c|}{1111000} &
\multicolumn{2}{c|}{00000} &
\multicolumn{1}{c|}{rs1} &
\multicolumn{1}{c|}{000} &
\multicolumn{1}{c|}{rd} &
\multicolumn{1}{c|}{1010011} & FMV.W.X \\
\cline{2-11}
  

&
\multicolumn{10}{c}{} & \\
&
\multicolumn{10}{c}{\bf RV64F Standard Extension (in addition to RV32F)} & \\
\cline{2-11}
  

&
\multicolumn{4}{|c|}{1100000} &
\multicolumn{2}{c|}{00010} &
\multicolumn{1}{c|}{rs1} &
\multicolumn{1}{c|}{rm} &
\multicolumn{1}{c|}{rd} &
\multicolumn{1}{c|}{1010011} & FCVT.L.S \\
\cline{2-11}
  

&
\multicolumn{4}{|c|}{1100000} &
\multicolumn{2}{c|}{00011} &
\multicolumn{1}{c|}{rs1} &
\multicolumn{1}{c|}{rm} &
\multicolumn{1}{c|}{rd} &
\multicolumn{1}{c|}{1010011} & FCVT.LU.S \\
\cline{2-11}
  

&
\multicolumn{4}{|c|}{1101000} &
\multicolumn{2}{c|}{00010} &
\multicolumn{1}{c|}{rs1} &
\multicolumn{1}{c|}{rm} &
\multicolumn{1}{c|}{rd} &
\multicolumn{1}{c|}{1010011} & FCVT.S.L \\
\cline{2-11}
  

&
\multicolumn{4}{|c|}{1101000} &
\multicolumn{2}{c|}{00011} &
\multicolumn{1}{c|}{rs1} &
\multicolumn{1}{c|}{rm} &
\multicolumn{1}{c|}{rd} &
\multicolumn{1}{c|}{1010011} & FCVT.S.LU \\
\cline{2-11}
  

\end{tabular}
\end{center}
\end{small}

\end{table}
  

\newpage

\begin{table}[p]
\begin{small}
\begin{center}
\begin{tabular}{p{0in}p{0.4in}p{0.05in}p{0.05in}p{0.05in}p{0.05in}p{0.4in}p{0.6in}p{0.4in}p{0.6in}p{0.7in}l}
& & & & & & & & & & \\
                      &
\multicolumn{1}{l}{\instbit{31}} &
\multicolumn{1}{r}{\instbit{27}} &
\instbit{26} &
\instbit{25} &
\multicolumn{1}{l}{\instbit{24}} &
\multicolumn{1}{r}{\instbit{20}} &
\instbitrange{19}{15} &
\instbitrange{14}{12} &
\instbitrange{11}{7} &
\instbitrange{6}{0} \\
\cline{2-11}


&
\multicolumn{4}{|c|}{funct7} &
\multicolumn{2}{c|}{rs2} &
\multicolumn{1}{c|}{rs1} &
\multicolumn{1}{c|}{funct3} &
\multicolumn{1}{c|}{rd} &
\multicolumn{1}{c|}{opcode} & R-type \\
\cline{2-11}


&
\multicolumn{2}{|c|}{rs3} &
\multicolumn{2}{c|}{funct2} &
\multicolumn{2}{c|}{rs2} &
\multicolumn{1}{c|}{rs1} &
\multicolumn{1}{c|}{funct3} &
\multicolumn{1}{c|}{rd} &
\multicolumn{1}{c|}{opcode} & R4-type \\
\cline{2-11}
  

&
\multicolumn{6}{|c|}{imm[11:0]} &
\multicolumn{1}{c|}{rs1} &
\multicolumn{1}{c|}{funct3} &
\multicolumn{1}{c|}{rd} &
\multicolumn{1}{c|}{opcode} & I-type \\
\cline{2-11}


&
\multicolumn{4}{|c|}{imm[11:5]} &
\multicolumn{2}{c|}{rs2} &
\multicolumn{1}{c|}{rs1} &
\multicolumn{1}{c|}{funct3} &
\multicolumn{1}{c|}{imm[4:0]} &
\multicolumn{1}{c|}{opcode} & S-type \\
\cline{2-11}


&
\multicolumn{10}{c}{} & \\
&
\multicolumn{10}{c}{\bf RV32D Standard Extension} & \\
\cline{2-11}
  

&
\multicolumn{6}{|c|}{imm[11:0]} &
\multicolumn{1}{c|}{rs1} &
\multicolumn{1}{c|}{011} &
\multicolumn{1}{c|}{rd} &
\multicolumn{1}{c|}{0000111} & FLD \\
\cline{2-11}
  

&
\multicolumn{4}{|c|}{imm[11:5]} &
\multicolumn{2}{c|}{rs2} &
\multicolumn{1}{c|}{rs1} &
\multicolumn{1}{c|}{011} &
\multicolumn{1}{c|}{imm[4:0]} &
\multicolumn{1}{c|}{0100111} & FSD \\
\cline{2-11}
  

&
\multicolumn{2}{|c|}{rs3} &
\multicolumn{2}{c|}{01} &
\multicolumn{2}{c|}{rs2} &
\multicolumn{1}{c|}{rs1} &
\multicolumn{1}{c|}{rm} &
\multicolumn{1}{c|}{rd} &
\multicolumn{1}{c|}{1000011} & FMADD.D \\
\cline{2-11}
  

&
\multicolumn{2}{|c|}{rs3} &
\multicolumn{2}{c|}{01} &
\multicolumn{2}{c|}{rs2} &
\multicolumn{1}{c|}{rs1} &
\multicolumn{1}{c|}{rm} &
\multicolumn{1}{c|}{rd} &
\multicolumn{1}{c|}{1000111} & FMSUB.D \\
\cline{2-11}
  

&
\multicolumn{2}{|c|}{rs3} &
\multicolumn{2}{c|}{01} &
\multicolumn{2}{c|}{rs2} &
\multicolumn{1}{c|}{rs1} &
\multicolumn{1}{c|}{rm} &
\multicolumn{1}{c|}{rd} &
\multicolumn{1}{c|}{1001011} & FNMSUB.D \\
\cline{2-11}
  

&
\multicolumn{2}{|c|}{rs3} &
\multicolumn{2}{c|}{01} &
\multicolumn{2}{c|}{rs2} &
\multicolumn{1}{c|}{rs1} &
\multicolumn{1}{c|}{rm} &
\multicolumn{1}{c|}{rd} &
\multicolumn{1}{c|}{1001111} & FNMADD.D \\
\cline{2-11}
  

&
\multicolumn{4}{|c|}{0000001} &
\multicolumn{2}{c|}{rs2} &
\multicolumn{1}{c|}{rs1} &
\multicolumn{1}{c|}{rm} &
\multicolumn{1}{c|}{rd} &
\multicolumn{1}{c|}{1010011} & FADD.D \\
\cline{2-11}
  

&
\multicolumn{4}{|c|}{0000101} &
\multicolumn{2}{c|}{rs2} &
\multicolumn{1}{c|}{rs1} &
\multicolumn{1}{c|}{rm} &
\multicolumn{1}{c|}{rd} &
\multicolumn{1}{c|}{1010011} & FSUB.D \\
\cline{2-11}
  

&
\multicolumn{4}{|c|}{0001001} &
\multicolumn{2}{c|}{rs2} &
\multicolumn{1}{c|}{rs1} &
\multicolumn{1}{c|}{rm} &
\multicolumn{1}{c|}{rd} &
\multicolumn{1}{c|}{1010011} & FMUL.D \\
\cline{2-11}
  

&
\multicolumn{4}{|c|}{0001101} &
\multicolumn{2}{c|}{rs2} &
\multicolumn{1}{c|}{rs1} &
\multicolumn{1}{c|}{rm} &
\multicolumn{1}{c|}{rd} &
\multicolumn{1}{c|}{1010011} & FDIV.D \\
\cline{2-11}
  

&
\multicolumn{4}{|c|}{0101101} &
\multicolumn{2}{c|}{00000} &
\multicolumn{1}{c|}{rs1} &
\multicolumn{1}{c|}{rm} &
\multicolumn{1}{c|}{rd} &
\multicolumn{1}{c|}{1010011} & FSQRT.D \\
\cline{2-11}
  

&
\multicolumn{4}{|c|}{0010001} &
\multicolumn{2}{c|}{rs2} &
\multicolumn{1}{c|}{rs1} &
\multicolumn{1}{c|}{000} &
\multicolumn{1}{c|}{rd} &
\multicolumn{1}{c|}{1010011} & FSGNJ.D \\
\cline{2-11}
  

&
\multicolumn{4}{|c|}{0010001} &
\multicolumn{2}{c|}{rs2} &
\multicolumn{1}{c|}{rs1} &
\multicolumn{1}{c|}{001} &
\multicolumn{1}{c|}{rd} &
\multicolumn{1}{c|}{1010011} & FSGNJN.D \\
\cline{2-11}
  

&
\multicolumn{4}{|c|}{0010001} &
\multicolumn{2}{c|}{rs2} &
\multicolumn{1}{c|}{rs1} &
\multicolumn{1}{c|}{010} &
\multicolumn{1}{c|}{rd} &
\multicolumn{1}{c|}{1010011} & FSGNJX.D \\
\cline{2-11}
  

&
\multicolumn{4}{|c|}{0010101} &
\multicolumn{2}{c|}{rs2} &
\multicolumn{1}{c|}{rs1} &
\multicolumn{1}{c|}{000} &
\multicolumn{1}{c|}{rd} &
\multicolumn{1}{c|}{1010011} & FMIN.D \\
\cline{2-11}
  

&
\multicolumn{4}{|c|}{0010101} &
\multicolumn{2}{c|}{rs2} &
\multicolumn{1}{c|}{rs1} &
\multicolumn{1}{c|}{001} &
\multicolumn{1}{c|}{rd} &
\multicolumn{1}{c|}{1010011} & FMAX.D \\
\cline{2-11}
  

&
\multicolumn{4}{|c|}{0100000} &
\multicolumn{2}{c|}{00001} &
\multicolumn{1}{c|}{rs1} &
\multicolumn{1}{c|}{rm} &
\multicolumn{1}{c|}{rd} &
\multicolumn{1}{c|}{1010011} & FCVT.S.D \\
\cline{2-11}
  

&
\multicolumn{4}{|c|}{0100001} &
\multicolumn{2}{c|}{00000} &
\multicolumn{1}{c|}{rs1} &
\multicolumn{1}{c|}{rm} &
\multicolumn{1}{c|}{rd} &
\multicolumn{1}{c|}{1010011} & FCVT.D.S \\
\cline{2-11}
  

&
\multicolumn{4}{|c|}{1010001} &
\multicolumn{2}{c|}{rs2} &
\multicolumn{1}{c|}{rs1} &
\multicolumn{1}{c|}{010} &
\multicolumn{1}{c|}{rd} &
\multicolumn{1}{c|}{1010011} & FEQ.D \\
\cline{2-11}
  

&
\multicolumn{4}{|c|}{1010001} &
\multicolumn{2}{c|}{rs2} &
\multicolumn{1}{c|}{rs1} &
\multicolumn{1}{c|}{001} &
\multicolumn{1}{c|}{rd} &
\multicolumn{1}{c|}{1010011} & FLT.D \\
\cline{2-11}
  

&
\multicolumn{4}{|c|}{1010001} &
\multicolumn{2}{c|}{rs2} &
\multicolumn{1}{c|}{rs1} &
\multicolumn{1}{c|}{000} &
\multicolumn{1}{c|}{rd} &
\multicolumn{1}{c|}{1010011} & FLE.D \\
\cline{2-11}
  

&
\multicolumn{4}{|c|}{1110001} &
\multicolumn{2}{c|}{00000} &
\multicolumn{1}{c|}{rs1} &
\multicolumn{1}{c|}{001} &
\multicolumn{1}{c|}{rd} &
\multicolumn{1}{c|}{1010011} & FCLASS.D \\
\cline{2-11}
  

&
\multicolumn{4}{|c|}{1100001} &
\multicolumn{2}{c|}{00000} &
\multicolumn{1}{c|}{rs1} &
\multicolumn{1}{c|}{rm} &
\multicolumn{1}{c|}{rd} &
\multicolumn{1}{c|}{1010011} & FCVT.W.D \\
\cline{2-11}
  

&
\multicolumn{4}{|c|}{1100001} &
\multicolumn{2}{c|}{00001} &
\multicolumn{1}{c|}{rs1} &
\multicolumn{1}{c|}{rm} &
\multicolumn{1}{c|}{rd} &
\multicolumn{1}{c|}{1010011} & FCVT.WU.D \\
\cline{2-11}
  

&
\multicolumn{4}{|c|}{1101001} &
\multicolumn{2}{c|}{00000} &
\multicolumn{1}{c|}{rs1} &
\multicolumn{1}{c|}{rm} &
\multicolumn{1}{c|}{rd} &
\multicolumn{1}{c|}{1010011} & FCVT.D.W \\
\cline{2-11}
  

&
\multicolumn{4}{|c|}{1101001} &
\multicolumn{2}{c|}{00001} &
\multicolumn{1}{c|}{rs1} &
\multicolumn{1}{c|}{rm} &
\multicolumn{1}{c|}{rd} &
\multicolumn{1}{c|}{1010011} & FCVT.D.WU \\
\cline{2-11}
  

&
\multicolumn{10}{c}{} & \\
&
\multicolumn{10}{c}{\bf RV64D Standard Extension (in addition to RV32D)} & \\
\cline{2-11}
  

&
\multicolumn{4}{|c|}{1100001} &
\multicolumn{2}{c|}{00010} &
\multicolumn{1}{c|}{rs1} &
\multicolumn{1}{c|}{rm} &
\multicolumn{1}{c|}{rd} &
\multicolumn{1}{c|}{1010011} & FCVT.L.D \\
\cline{2-11}
  

&
\multicolumn{4}{|c|}{1100001} &
\multicolumn{2}{c|}{00011} &
\multicolumn{1}{c|}{rs1} &
\multicolumn{1}{c|}{rm} &
\multicolumn{1}{c|}{rd} &
\multicolumn{1}{c|}{1010011} & FCVT.LU.D \\
\cline{2-11}
  

&
\multicolumn{4}{|c|}{1110001} &
\multicolumn{2}{c|}{00000} &
\multicolumn{1}{c|}{rs1} &
\multicolumn{1}{c|}{000} &
\multicolumn{1}{c|}{rd} &
\multicolumn{1}{c|}{1010011} & FMV.X.D \\
\cline{2-11}
  

&
\multicolumn{4}{|c|}{1101001} &
\multicolumn{2}{c|}{00010} &
\multicolumn{1}{c|}{rs1} &
\multicolumn{1}{c|}{rm} &
\multicolumn{1}{c|}{rd} &
\multicolumn{1}{c|}{1010011} & FCVT.D.L \\
\cline{2-11}
  

&
\multicolumn{4}{|c|}{1101001} &
\multicolumn{2}{c|}{00011} &
\multicolumn{1}{c|}{rs1} &
\multicolumn{1}{c|}{rm} &
\multicolumn{1}{c|}{rd} &
\multicolumn{1}{c|}{1010011} & FCVT.D.LU \\
\cline{2-11}
  

&
\multicolumn{4}{|c|}{1111001} &
\multicolumn{2}{c|}{00000} &
\multicolumn{1}{c|}{rs1} &
\multicolumn{1}{c|}{000} &
\multicolumn{1}{c|}{rd} &
\multicolumn{1}{c|}{1010011} & FMV.D.X \\
\cline{2-11}
  

\end{tabular}
\end{center}
\end{small}

\end{table}
  

\newpage

\begin{table}[p]
\begin{small}
\begin{center}
\begin{tabular}{p{0in}p{0.4in}p{0.05in}p{0.05in}p{0.05in}p{0.05in}p{0.4in}p{0.6in}p{0.4in}p{0.6in}p{0.7in}l}
& & & & & & & & & & \\
                      &
\multicolumn{1}{l}{\instbit{31}} &
\multicolumn{1}{r}{\instbit{27}} &
\instbit{26} &
\instbit{25} &
\multicolumn{1}{l}{\instbit{24}} &
\multicolumn{1}{r}{\instbit{20}} &
\instbitrange{19}{15} &
\instbitrange{14}{12} &
\instbitrange{11}{7} &
\instbitrange{6}{0} \\
\cline{2-11}


&
\multicolumn{4}{|c|}{funct7} &
\multicolumn{2}{c|}{rs2} &
\multicolumn{1}{c|}{rs1} &
\multicolumn{1}{c|}{funct3} &
\multicolumn{1}{c|}{rd} &
\multicolumn{1}{c|}{opcode} & R-type \\
\cline{2-11}


&
\multicolumn{2}{|c|}{rs3} &
\multicolumn{2}{c|}{funct2} &
\multicolumn{2}{c|}{rs2} &
\multicolumn{1}{c|}{rs1} &
\multicolumn{1}{c|}{funct3} &
\multicolumn{1}{c|}{rd} &
\multicolumn{1}{c|}{opcode} & R4-type \\
\cline{2-11}
  

&
\multicolumn{6}{|c|}{imm[11:0]} &
\multicolumn{1}{c|}{rs1} &
\multicolumn{1}{c|}{funct3} &
\multicolumn{1}{c|}{rd} &
\multicolumn{1}{c|}{opcode} & I-type \\
\cline{2-11}


&
\multicolumn{4}{|c|}{imm[11:5]} &
\multicolumn{2}{c|}{rs2} &
\multicolumn{1}{c|}{rs1} &
\multicolumn{1}{c|}{funct3} &
\multicolumn{1}{c|}{imm[4:0]} &
\multicolumn{1}{c|}{opcode} & S-type \\
\cline{2-11}


&
\multicolumn{10}{c}{} & \\
&
\multicolumn{10}{c}{\bf RV32Q Standard Extension} & \\
\cline{2-11}
  

&
\multicolumn{6}{|c|}{imm[11:0]} &
\multicolumn{1}{c|}{rs1} &
\multicolumn{1}{c|}{100} &
\multicolumn{1}{c|}{rd} &
\multicolumn{1}{c|}{0000111} & FLQ \\
\cline{2-11}
  

&
\multicolumn{4}{|c|}{imm[11:5]} &
\multicolumn{2}{c|}{rs2} &
\multicolumn{1}{c|}{rs1} &
\multicolumn{1}{c|}{100} &
\multicolumn{1}{c|}{imm[4:0]} &
\multicolumn{1}{c|}{0100111} & FSQ \\
\cline{2-11}
  

&
\multicolumn{2}{|c|}{rs3} &
\multicolumn{2}{c|}{11} &
\multicolumn{2}{c|}{rs2} &
\multicolumn{1}{c|}{rs1} &
\multicolumn{1}{c|}{rm} &
\multicolumn{1}{c|}{rd} &
\multicolumn{1}{c|}{1000011} & FMADD.Q \\
\cline{2-11}
  

&
\multicolumn{2}{|c|}{rs3} &
\multicolumn{2}{c|}{11} &
\multicolumn{2}{c|}{rs2} &
\multicolumn{1}{c|}{rs1} &
\multicolumn{1}{c|}{rm} &
\multicolumn{1}{c|}{rd} &
\multicolumn{1}{c|}{1000111} & FMSUB.Q \\
\cline{2-11}
  

&
\multicolumn{2}{|c|}{rs3} &
\multicolumn{2}{c|}{11} &
\multicolumn{2}{c|}{rs2} &
\multicolumn{1}{c|}{rs1} &
\multicolumn{1}{c|}{rm} &
\multicolumn{1}{c|}{rd} &
\multicolumn{1}{c|}{1001011} & FNMSUB.Q \\
\cline{2-11}
  

&
\multicolumn{2}{|c|}{rs3} &
\multicolumn{2}{c|}{11} &
\multicolumn{2}{c|}{rs2} &
\multicolumn{1}{c|}{rs1} &
\multicolumn{1}{c|}{rm} &
\multicolumn{1}{c|}{rd} &
\multicolumn{1}{c|}{1001111} & FNMADD.Q \\
\cline{2-11}
  

&
\multicolumn{4}{|c|}{0000011} &
\multicolumn{2}{c|}{rs2} &
\multicolumn{1}{c|}{rs1} &
\multicolumn{1}{c|}{rm} &
\multicolumn{1}{c|}{rd} &
\multicolumn{1}{c|}{1010011} & FADD.Q \\
\cline{2-11}
  

&
\multicolumn{4}{|c|}{0000111} &
\multicolumn{2}{c|}{rs2} &
\multicolumn{1}{c|}{rs1} &
\multicolumn{1}{c|}{rm} &
\multicolumn{1}{c|}{rd} &
\multicolumn{1}{c|}{1010011} & FSUB.Q \\
\cline{2-11}
  

&
\multicolumn{4}{|c|}{0001011} &
\multicolumn{2}{c|}{rs2} &
\multicolumn{1}{c|}{rs1} &
\multicolumn{1}{c|}{rm} &
\multicolumn{1}{c|}{rd} &
\multicolumn{1}{c|}{1010011} & FMUL.Q \\
\cline{2-11}
  

&
\multicolumn{4}{|c|}{0001111} &
\multicolumn{2}{c|}{rs2} &
\multicolumn{1}{c|}{rs1} &
\multicolumn{1}{c|}{rm} &
\multicolumn{1}{c|}{rd} &
\multicolumn{1}{c|}{1010011} & FDIV.Q \\
\cline{2-11}
  

&
\multicolumn{4}{|c|}{0101111} &
\multicolumn{2}{c|}{00000} &
\multicolumn{1}{c|}{rs1} &
\multicolumn{1}{c|}{rm} &
\multicolumn{1}{c|}{rd} &
\multicolumn{1}{c|}{1010011} & FSQRT.Q \\
\cline{2-11}
  

&
\multicolumn{4}{|c|}{0010011} &
\multicolumn{2}{c|}{rs2} &
\multicolumn{1}{c|}{rs1} &
\multicolumn{1}{c|}{000} &
\multicolumn{1}{c|}{rd} &
\multicolumn{1}{c|}{1010011} & FSGNJ.Q \\
\cline{2-11}
  

&
\multicolumn{4}{|c|}{0010011} &
\multicolumn{2}{c|}{rs2} &
\multicolumn{1}{c|}{rs1} &
\multicolumn{1}{c|}{001} &
\multicolumn{1}{c|}{rd} &
\multicolumn{1}{c|}{1010011} & FSGNJN.Q \\
\cline{2-11}
  

&
\multicolumn{4}{|c|}{0010011} &
\multicolumn{2}{c|}{rs2} &
\multicolumn{1}{c|}{rs1} &
\multicolumn{1}{c|}{010} &
\multicolumn{1}{c|}{rd} &
\multicolumn{1}{c|}{1010011} & FSGNJX.Q \\
\cline{2-11}
  

&
\multicolumn{4}{|c|}{0010111} &
\multicolumn{2}{c|}{rs2} &
\multicolumn{1}{c|}{rs1} &
\multicolumn{1}{c|}{000} &
\multicolumn{1}{c|}{rd} &
\multicolumn{1}{c|}{1010011} & FMIN.Q \\
\cline{2-11}
  

&
\multicolumn{4}{|c|}{0010111} &
\multicolumn{2}{c|}{rs2} &
\multicolumn{1}{c|}{rs1} &
\multicolumn{1}{c|}{001} &
\multicolumn{1}{c|}{rd} &
\multicolumn{1}{c|}{1010011} & FMAX.Q \\
\cline{2-11}
  

&
\multicolumn{4}{|c|}{0100000} &
\multicolumn{2}{c|}{00011} &
\multicolumn{1}{c|}{rs1} &
\multicolumn{1}{c|}{rm} &
\multicolumn{1}{c|}{rd} &
\multicolumn{1}{c|}{1010011} & FCVT.S.Q \\
\cline{2-11}
  

&
\multicolumn{4}{|c|}{0100011} &
\multicolumn{2}{c|}{00000} &
\multicolumn{1}{c|}{rs1} &
\multicolumn{1}{c|}{rm} &
\multicolumn{1}{c|}{rd} &
\multicolumn{1}{c|}{1010011} & FCVT.Q.S \\
\cline{2-11}
  

&
\multicolumn{4}{|c|}{0100001} &
\multicolumn{2}{c|}{00011} &
\multicolumn{1}{c|}{rs1} &
\multicolumn{1}{c|}{rm} &
\multicolumn{1}{c|}{rd} &
\multicolumn{1}{c|}{1010011} & FCVT.D.Q \\
\cline{2-11}
  

&
\multicolumn{4}{|c|}{0100011} &
\multicolumn{2}{c|}{00001} &
\multicolumn{1}{c|}{rs1} &
\multicolumn{1}{c|}{rm} &
\multicolumn{1}{c|}{rd} &
\multicolumn{1}{c|}{1010011} & FCVT.Q.D \\
\cline{2-11}
  

&
\multicolumn{4}{|c|}{1010011} &
\multicolumn{2}{c|}{rs2} &
\multicolumn{1}{c|}{rs1} &
\multicolumn{1}{c|}{010} &
\multicolumn{1}{c|}{rd} &
\multicolumn{1}{c|}{1010011} & FEQ.Q \\
\cline{2-11}
  

&
\multicolumn{4}{|c|}{1010011} &
\multicolumn{2}{c|}{rs2} &
\multicolumn{1}{c|}{rs1} &
\multicolumn{1}{c|}{001} &
\multicolumn{1}{c|}{rd} &
\multicolumn{1}{c|}{1010011} & FLT.Q \\
\cline{2-11}
  

&
\multicolumn{4}{|c|}{1010011} &
\multicolumn{2}{c|}{rs2} &
\multicolumn{1}{c|}{rs1} &
\multicolumn{1}{c|}{000} &
\multicolumn{1}{c|}{rd} &
\multicolumn{1}{c|}{1010011} & FLE.Q \\
\cline{2-11}
  

&
\multicolumn{4}{|c|}{1110011} &
\multicolumn{2}{c|}{00000} &
\multicolumn{1}{c|}{rs1} &
\multicolumn{1}{c|}{001} &
\multicolumn{1}{c|}{rd} &
\multicolumn{1}{c|}{1010011} & FCLASS.Q \\
\cline{2-11}
  

&
\multicolumn{4}{|c|}{1100011} &
\multicolumn{2}{c|}{00000} &
\multicolumn{1}{c|}{rs1} &
\multicolumn{1}{c|}{rm} &
\multicolumn{1}{c|}{rd} &
\multicolumn{1}{c|}{1010011} & FCVT.W.Q \\
\cline{2-11}
  

&
\multicolumn{4}{|c|}{1100011} &
\multicolumn{2}{c|}{00001} &
\multicolumn{1}{c|}{rs1} &
\multicolumn{1}{c|}{rm} &
\multicolumn{1}{c|}{rd} &
\multicolumn{1}{c|}{1010011} & FCVT.WU.Q \\
\cline{2-11}
  

&
\multicolumn{4}{|c|}{1101011} &
\multicolumn{2}{c|}{00000} &
\multicolumn{1}{c|}{rs1} &
\multicolumn{1}{c|}{rm} &
\multicolumn{1}{c|}{rd} &
\multicolumn{1}{c|}{1010011} & FCVT.Q.W \\
\cline{2-11}
  

&
\multicolumn{4}{|c|}{1101011} &
\multicolumn{2}{c|}{00001} &
\multicolumn{1}{c|}{rs1} &
\multicolumn{1}{c|}{rm} &
\multicolumn{1}{c|}{rd} &
\multicolumn{1}{c|}{1010011} & FCVT.Q.WU \\
\cline{2-11}
  

&
\multicolumn{10}{c}{} & \\
&
\multicolumn{10}{c}{\bf RV64Q Standard Extension (in addition to RV32Q)} & \\
\cline{2-11}
  

&
\multicolumn{4}{|c|}{1100011} &
\multicolumn{2}{c|}{00010} &
\multicolumn{1}{c|}{rs1} &
\multicolumn{1}{c|}{rm} &
\multicolumn{1}{c|}{rd} &
\multicolumn{1}{c|}{1010011} & FCVT.L.Q \\
\cline{2-11}
  

&
\multicolumn{4}{|c|}{1100011} &
\multicolumn{2}{c|}{00011} &
\multicolumn{1}{c|}{rs1} &
\multicolumn{1}{c|}{rm} &
\multicolumn{1}{c|}{rd} &
\multicolumn{1}{c|}{1010011} & FCVT.LU.Q \\
\cline{2-11}
  

&
\multicolumn{4}{|c|}{1101011} &
\multicolumn{2}{c|}{00010} &
\multicolumn{1}{c|}{rs1} &
\multicolumn{1}{c|}{rm} &
\multicolumn{1}{c|}{rd} &
\multicolumn{1}{c|}{1010011} & FCVT.Q.L \\
\cline{2-11}
  

&
\multicolumn{4}{|c|}{1101011} &
\multicolumn{2}{c|}{00011} &
\multicolumn{1}{c|}{rs1} &
\multicolumn{1}{c|}{rm} &
\multicolumn{1}{c|}{rd} &
\multicolumn{1}{c|}{1010011} & FCVT.Q.LU \\
\cline{2-11}

\end{tabular}
\end{center}
\end{small}

\end{table}



\newpage

\begin{table}[p]
\begin{small}
\begin{center}
\begin{tabular}{p{0in}p{0.4in}p{0.05in}p{0.05in}p{0.05in}p{0.05in}p{0.4in}p{0.6in}p{0.4in}p{0.6in}p{0.7in}l}
& & & & & & & & & & \\
                      &
\multicolumn{1}{l}{\instbit{31}} &
\multicolumn{1}{r}{\instbit{27}} &
\instbit{26} &
\instbit{25} &
\multicolumn{1}{l}{\instbit{24}} &
\multicolumn{1}{r}{\instbit{20}} &
\instbitrange{19}{15} &
\instbitrange{14}{12} &
\instbitrange{11}{7} &
\instbitrange{6}{0} \\
\cline{2-11}


&
\multicolumn{4}{|c|}{funct7} &
\multicolumn{2}{c|}{rs2} &
\multicolumn{1}{c|}{rs1} &
\multicolumn{1}{c|}{funct3} &
\multicolumn{1}{c|}{rd} &
\multicolumn{1}{c|}{opcode} & R-type \\
\cline{2-11}


&
\multicolumn{2}{|c|}{rs3} &
\multicolumn{2}{c|}{funct2} &
\multicolumn{2}{c|}{rs2} &
\multicolumn{1}{c|}{rs1} &
\multicolumn{1}{c|}{funct3} &
\multicolumn{1}{c|}{rd} &
\multicolumn{1}{c|}{opcode} & R4-type \\
\cline{2-11}


&
\multicolumn{6}{|c|}{imm[11:0]} &
\multicolumn{1}{c|}{rs1} &
\multicolumn{1}{c|}{funct3} &
\multicolumn{1}{c|}{rd} &
\multicolumn{1}{c|}{opcode} & I-type \\
\cline{2-11}


&
\multicolumn{4}{|c|}{imm[11:5]} &
\multicolumn{2}{c|}{rs2} &
\multicolumn{1}{c|}{rs1} &
\multicolumn{1}{c|}{funct3} &
\multicolumn{1}{c|}{imm[4:0]} &
\multicolumn{1}{c|}{opcode} & S-type \\
\cline{2-11}



&
\multicolumn{10}{c}{} & \\
&
\multicolumn{10}{c}{\bf RV32Zfh Standard Extension} & \\
\cline{2-11}


&
\multicolumn{6}{|c|}{imm[11:0]} &
\multicolumn{1}{c|}{rs1} &
\multicolumn{1}{c|}{001} &
\multicolumn{1}{c|}{rd} &
\multicolumn{1}{c|}{0000111} & FLH \\
\cline{2-11}


&
\multicolumn{4}{|c|}{imm[11:5]} &
\multicolumn{2}{c|}{rs2} &
\multicolumn{1}{c|}{rs1} &
\multicolumn{1}{c|}{001} &
\multicolumn{1}{c|}{imm[4:0]} &
\multicolumn{1}{c|}{0100111} & FSH \\
\cline{2-11}


&
\multicolumn{2}{|c|}{rs3} &
\multicolumn{2}{c|}{10} &
\multicolumn{2}{c|}{rs2} &
\multicolumn{1}{c|}{rs1} &
\multicolumn{1}{c|}{rm} &
\multicolumn{1}{c|}{rd} &
\multicolumn{1}{c|}{1000011} & FMADD.H \\
\cline{2-11}


&
\multicolumn{2}{|c|}{rs3} &
\multicolumn{2}{c|}{10} &
\multicolumn{2}{c|}{rs2} &
\multicolumn{1}{c|}{rs1} &
\multicolumn{1}{c|}{rm} &
\multicolumn{1}{c|}{rd} &
\multicolumn{1}{c|}{1000111} & FMSUB.H \\
\cline{2-11}


&
\multicolumn{2}{|c|}{rs3} &
\multicolumn{2}{c|}{10} &
\multicolumn{2}{c|}{rs2} &
\multicolumn{1}{c|}{rs1} &
\multicolumn{1}{c|}{rm} &
\multicolumn{1}{c|}{rd} &
\multicolumn{1}{c|}{1001011} & FNMSUB.H \\
\cline{2-11}


&
\multicolumn{2}{|c|}{rs3} &
\multicolumn{2}{c|}{10} &
\multicolumn{2}{c|}{rs2} &
\multicolumn{1}{c|}{rs1} &
\multicolumn{1}{c|}{rm} &
\multicolumn{1}{c|}{rd} &
\multicolumn{1}{c|}{1001111} & FNMADD.H \\
\cline{2-11}


&
\multicolumn{4}{|c|}{0000010} &
\multicolumn{2}{c|}{rs2} &
\multicolumn{1}{c|}{rs1} &
\multicolumn{1}{c|}{rm} &
\multicolumn{1}{c|}{rd} &
\multicolumn{1}{c|}{1010011} & FADD.H \\
\cline{2-11}


&
\multicolumn{4}{|c|}{0000110} &
\multicolumn{2}{c|}{rs2} &
\multicolumn{1}{c|}{rs1} &
\multicolumn{1}{c|}{rm} &
\multicolumn{1}{c|}{rd} &
\multicolumn{1}{c|}{1010011} & FSUB.H \\
\cline{2-11}


&
\multicolumn{4}{|c|}{0001010} &
\multicolumn{2}{c|}{rs2} &
\multicolumn{1}{c|}{rs1} &
\multicolumn{1}{c|}{rm} &
\multicolumn{1}{c|}{rd} &
\multicolumn{1}{c|}{1010011} & FMUL.H \\
\cline{2-11}


&
\multicolumn{4}{|c|}{0001110} &
\multicolumn{2}{c|}{rs2} &
\multicolumn{1}{c|}{rs1} &
\multicolumn{1}{c|}{rm} &
\multicolumn{1}{c|}{rd} &
\multicolumn{1}{c|}{1010011} & FDIV.H \\
\cline{2-11}


&
\multicolumn{4}{|c|}{0101110} &
\multicolumn{2}{c|}{00000} &
\multicolumn{1}{c|}{rs1} &
\multicolumn{1}{c|}{rm} &
\multicolumn{1}{c|}{rd} &
\multicolumn{1}{c|}{1010011} & FSQRT.H \\
\cline{2-11}


&
\multicolumn{4}{|c|}{0010010} &
\multicolumn{2}{c|}{rs2} &
\multicolumn{1}{c|}{rs1} &
\multicolumn{1}{c|}{000} &
\multicolumn{1}{c|}{rd} &
\multicolumn{1}{c|}{1010011} & FSGNJ.H \\
\cline{2-11}


&
\multicolumn{4}{|c|}{0010010} &
\multicolumn{2}{c|}{rs2} &
\multicolumn{1}{c|}{rs1} &
\multicolumn{1}{c|}{001} &
\multicolumn{1}{c|}{rd} &
\multicolumn{1}{c|}{1010011} & FSGNJN.H \\
\cline{2-11}


&
\multicolumn{4}{|c|}{0010010} &
\multicolumn{2}{c|}{rs2} &
\multicolumn{1}{c|}{rs1} &
\multicolumn{1}{c|}{010} &
\multicolumn{1}{c|}{rd} &
\multicolumn{1}{c|}{1010011} & FSGNJX.H \\
\cline{2-11}


&
\multicolumn{4}{|c|}{0010110} &
\multicolumn{2}{c|}{rs2} &
\multicolumn{1}{c|}{rs1} &
\multicolumn{1}{c|}{000} &
\multicolumn{1}{c|}{rd} &
\multicolumn{1}{c|}{1010011} & FMIN.H \\
\cline{2-11}


&
\multicolumn{4}{|c|}{0010110} &
\multicolumn{2}{c|}{rs2} &
\multicolumn{1}{c|}{rs1} &
\multicolumn{1}{c|}{001} &
\multicolumn{1}{c|}{rd} &
\multicolumn{1}{c|}{1010011} & FMAX.H \\
\cline{2-11}


&
\multicolumn{4}{|c|}{0100000} &
\multicolumn{2}{c|}{00010} &
\multicolumn{1}{c|}{rs1} &
\multicolumn{1}{c|}{rm} &
\multicolumn{1}{c|}{rd} &
\multicolumn{1}{c|}{1010011} & FCVT.S.H \\
\cline{2-11}


&
\multicolumn{4}{|c|}{0100001} &
\multicolumn{2}{c|}{00010} &
\multicolumn{1}{c|}{rs1} &
\multicolumn{1}{c|}{rm} &
\multicolumn{1}{c|}{rd} &
\multicolumn{1}{c|}{1010011} & FCVT.D.H \\
\cline{2-11}


&
\multicolumn{4}{|c|}{0100011} &
\multicolumn{2}{c|}{00010} &
\multicolumn{1}{c|}{rs1} &
\multicolumn{1}{c|}{rm} &
\multicolumn{1}{c|}{rd} &
\multicolumn{1}{c|}{1010011} & FCVT.Q.H \\
\cline{2-11}


&
\multicolumn{4}{|c|}{0100010} &
\multicolumn{2}{c|}{00000} &
\multicolumn{1}{c|}{rs1} &
\multicolumn{1}{c|}{rm} &
\multicolumn{1}{c|}{rd} &
\multicolumn{1}{c|}{1010011} & FCVT.H.S \\
\cline{2-11}


&
\multicolumn{4}{|c|}{0100010} &
\multicolumn{2}{c|}{00001} &
\multicolumn{1}{c|}{rs1} &
\multicolumn{1}{c|}{rm} &
\multicolumn{1}{c|}{rd} &
\multicolumn{1}{c|}{1010011} & FCVT.H.D \\
\cline{2-11}


&
\multicolumn{4}{|c|}{0100010} &
\multicolumn{2}{c|}{00011} &
\multicolumn{1}{c|}{rs1} &
\multicolumn{1}{c|}{rm} &
\multicolumn{1}{c|}{rd} &
\multicolumn{1}{c|}{1010011} & FCVT.H.Q \\
\cline{2-11}


&
\multicolumn{4}{|c|}{1110010} &
\multicolumn{2}{c|}{00000} &
\multicolumn{1}{c|}{rs1} &
\multicolumn{1}{c|}{000} &
\multicolumn{1}{c|}{rd} &
\multicolumn{1}{c|}{1010011} & FMV.X.H \\
\cline{2-11}


&
\multicolumn{4}{|c|}{1010010} &
\multicolumn{2}{c|}{rs2} &
\multicolumn{1}{c|}{rs1} &
\multicolumn{1}{c|}{010} &
\multicolumn{1}{c|}{rd} &
\multicolumn{1}{c|}{1010011} & FEQ.H \\
\cline{2-11}


&
\multicolumn{4}{|c|}{1010010} &
\multicolumn{2}{c|}{rs2} &
\multicolumn{1}{c|}{rs1} &
\multicolumn{1}{c|}{001} &
\multicolumn{1}{c|}{rd} &
\multicolumn{1}{c|}{1010011} & FLT.H \\
\cline{2-11}


&
\multicolumn{4}{|c|}{1010010} &
\multicolumn{2}{c|}{rs2} &
\multicolumn{1}{c|}{rs1} &
\multicolumn{1}{c|}{000} &
\multicolumn{1}{c|}{rd} &
\multicolumn{1}{c|}{1010011} & FLE.H \\
\cline{2-11}


&
\multicolumn{4}{|c|}{1110010} &
\multicolumn{2}{c|}{00000} &
\multicolumn{1}{c|}{rs1} &
\multicolumn{1}{c|}{001} &
\multicolumn{1}{c|}{rd} &
\multicolumn{1}{c|}{1010011} & FCLASS.H \\
\cline{2-11}


&
\multicolumn{4}{|c|}{1100010} &
\multicolumn{2}{c|}{00000} &
\multicolumn{1}{c|}{rs1} &
\multicolumn{1}{c|}{rm} &
\multicolumn{1}{c|}{rd} &
\multicolumn{1}{c|}{1010011} & FCVT.W.H \\
\cline{2-11}


&
\multicolumn{4}{|c|}{1100010} &
\multicolumn{2}{c|}{00001} &
\multicolumn{1}{c|}{rs1} &
\multicolumn{1}{c|}{rm} &
\multicolumn{1}{c|}{rd} &
\multicolumn{1}{c|}{1010011} & FCVT.WU.H \\
\cline{2-11}


&
\multicolumn{4}{|c|}{1101010} &
\multicolumn{2}{c|}{00000} &
\multicolumn{1}{c|}{rs1} &
\multicolumn{1}{c|}{rm} &
\multicolumn{1}{c|}{rd} &
\multicolumn{1}{c|}{1010011} & FCVT.H.W \\
\cline{2-11}


&
\multicolumn{4}{|c|}{1101010} &
\multicolumn{2}{c|}{00001} &
\multicolumn{1}{c|}{rs1} &
\multicolumn{1}{c|}{rm} &
\multicolumn{1}{c|}{rd} &
\multicolumn{1}{c|}{1010011} & FCVT.H.WU \\
\cline{2-11}


&
\multicolumn{4}{|c|}{1111010} &
\multicolumn{2}{c|}{00000} &
\multicolumn{1}{c|}{rs1} &
\multicolumn{1}{c|}{000} &
\multicolumn{1}{c|}{rd} &
\multicolumn{1}{c|}{1010011} & FMV.H.X \\
\cline{2-11}


&
\multicolumn{10}{c}{} & \\
&
\multicolumn{10}{c}{\bf RV64Zfh Standard Extension (in addition to RV32Zfh)} & \\
\cline{2-11}


&
\multicolumn{4}{|c|}{1100010} &
\multicolumn{2}{c|}{00010} &
\multicolumn{1}{c|}{rs1} &
\multicolumn{1}{c|}{rm} &
\multicolumn{1}{c|}{rd} &
\multicolumn{1}{c|}{1010011} & FCVT.L.H \\
\cline{2-11}


&
\multicolumn{4}{|c|}{1100010} &
\multicolumn{2}{c|}{00011} &
\multicolumn{1}{c|}{rs1} &
\multicolumn{1}{c|}{rm} &
\multicolumn{1}{c|}{rd} &
\multicolumn{1}{c|}{1010011} & FCVT.LU.H \\
\cline{2-11}


&
\multicolumn{4}{|c|}{1101010} &
\multicolumn{2}{c|}{00010} &
\multicolumn{1}{c|}{rs1} &
\multicolumn{1}{c|}{rm} &
\multicolumn{1}{c|}{rd} &
\multicolumn{1}{c|}{1010011} & FCVT.H.L \\
\cline{2-11}


&
\multicolumn{4}{|c|}{1101010} &
\multicolumn{2}{c|}{00011} &
\multicolumn{1}{c|}{rs1} &
\multicolumn{1}{c|}{rm} &
\multicolumn{1}{c|}{rd} &
\multicolumn{1}{c|}{1010011} & FCVT.H.LU \\
\cline{2-11}


\end{tabular}
\end{center}
\end{small}
\caption{Instruction listing for RISC-V}
\end{table}



\FloatBarrier
Table~\ref{rvgcsrnames} lists the CSRs that have
currently been allocated CSR addresses.  The timers, counters, and
floating-point CSRs are the only CSRs defined in this specification.

\begin{table}[htb!]
\begin{center}
\begin{tabular}{|l|l|l|l|}
\hline
Number    & Privilege & Name & Description \\
\hline
\multicolumn{4}{|c|}{Floating-Point Control and Status Registers} \\
\hline
\tt 0x001 & Read/write  &\tt fflags     & Floating-Point Accrued Exceptions. \\
\tt 0x002 & Read/write  &\tt frm        & Floating-Point Dynamic Rounding Mode. \\
\tt 0x003 & Read/write  &\tt fcsr       & Floating-Point Control and Status
Register ({\tt frm} + {\tt fflags}). \\
\hline
\multicolumn{4}{|c|}{Counters and Timers} \\
\hline
\tt 0xC00 & Read-only  &\tt cycle      & Cycle counter for RDCYCLE instruction. \\
\tt 0xC01 & Read-only  &\tt time       & Timer for RDTIME instruction. \\
\tt 0xC02 & Read-only  &\tt instret    & Instructions-retired counter for RDINSTRET instruction. \\
\tt 0xC80 & Read-only  &\tt cycleh     & Upper 32 bits of {\tt cycle}, RV32I only. \\
\tt 0xC81 & Read-only  &\tt timeh      & Upper 32 bits of {\tt time}, RV32I only. \\
\tt 0xC82 & Read-only  &\tt instreth   & Upper 32 bits of {\tt instret}, RV32I only. \\
\hline
\end{tabular}
\end{center}
\caption{RISC-V control and status register (CSR) address map.}
\label{rvgcsrnames}
\end{table}

